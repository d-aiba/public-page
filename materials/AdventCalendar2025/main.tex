%----------
%----------
%----------
\title{述語変換子意味論と余代数}
% \author{饗庭 大地 (Aiba, Daichi)}
\author{こっとん (\href{https://twitter.com/CottonShampoo}{@CottonShampoo}) }
\date{2025年12月20日 土曜日}
%----------
\documentclass[
		% book,
		head_space=20mm,
		foot_space=20mm,
		gutter=10mm,
		line_length=190mm
]{jlreq}
%----------
\input 0_preamble.tex
\usepackage{docmute} %ファイル分割
%----------
\usepackage{stmaryrd}

\begin{document}


\maketitle %タイトル

\vspace{2mm}

\begin{screen}
	\begin{center}
		\href{https://adventar.org/calendars/11397}{Mathematical Logic Advent Calendar 2025}の20日目の投稿です。
	\end{center}
% \end{screen}

% \vspace{4mm}

% \begin{screen}
	\begin{center}
		このドキュメントは,見やすく読み間違えにくいユニバーサルデザインフォントを採用しています。
	\end{center}
\end{screen}

\vspace{3mm}

\tableofcontents %目次


%----------
% \setcounter{part}{-1}
% \setcounter{chapter}{-1}
% \renewcommand{\thepart}{\arabic{part}}
% \renewcommand{\thesection}{\arabic{section}}
\setcounter{section}{-1}
%----------


\clearpage


\section{前置き}

\todo[inline]{Todo: \\ 1) ループ不変量の例を4つくらい追加で挙げる。 \\ 2) モナドとクライスリ圏の説明を簡潔に書く。 \\ 3) 最終章をちゃんと整理して書く。}


\subsection{この記事が想定する対象読者}


\subsection{この記事の目標}

この記事では、以下のことを目標とします。

\begin{framed}
	\noindent 大目標: %\\
	\begin{center} 不動点 (fixed points) こそが計算機科学における中心的概念であることを理解する! \end{center}
\end{framed}

小目標としては、以下の通りです。

\begin{framed}
	\noindent 小目標:
	\begin{enumerate}
		% \item 大目標: 不動点 (fixed points) こそが計算機科学における中心的概念であることを理解する!
		% \item 小目標 (1): 
		% \item 小目標 (2): 
		% \item 小目標 (3): 
		\item 
		\item 
		\item 
	\end{enumerate}
\end{framed}
% \vspace{3mm}

\subsection{この記事では扱わないこと}

本記事では、以下のことは扱いません。

\begin{itemize}
	\item 
	\item 
	\item 
\end{itemize}

また、以下の事柄については、既知のものとして仮定します。

\begin{itemize}
	\item 順序集合や不動点についての基本事項(ナスター・タルスキの定理、クリーネの不動点定理、など)
	\item 数理論理学における基本事項(古典命題論理、順序数の基本的な性質、様相論理のクリプキフレーム、など)
	\item 圏論の基本事項(圏の定義、関手、自然変換、極限と余極限、など)
\end{itemize}

% \vspace{5mm}

\clearpage

\section{ホーア論理と述語変換子}

\textbf{ホーア論理} (Hoare logic) と\textbf{述語変換子} (predicate transformer) は、プログラムの正しさ(意図した挙動をするか、バグがないかどうか)を数学的に検証するための、異なる2つのアプローチです。プログラムの満たしていてほしい性質(仕様)を論理式で記述し、それが満たされているかを論理学的手法を用いて確かめます。

たとえば、以下の例「入力 $N$ に対し、階乗 $N!$ を計算して出力するプログラム」を考えてみましょう。

\begin{multicols}{2}
% \noindent 
\begin{framed}
	\begin{verbatim}
		n := N
		k := 1
		while (n>0) {
				k := k*n;
				n := n-1;
		}
	\end{verbatim}
	~\\[-48pt]
\end{framed}
%
% \noindent 
\begin{framed}
\noindent 【このプログラムについて知りたいこと】
\begin{enumerate}
	\item 事後条件「$k=N!$」(出力 $k$ が入力 $N$ の階乗になる)を保証するには、何が分かればよいか?
	\item 事後条件「$k=N!$」を成り立たせるためには、最低限、どんな事前条件を課せばよいか?
\end{enumerate}
\end{framed}
\end{multicols}

$N$に正の自然数を入力してこのプログラムを実行した後に、\uwave{事後条件「$k=N!$」を満たしてほしい}ものとします。ループが何回実行されるかわからない中で、\uwave{あらゆる入力 $N$ に対してどうやって保証}すればよいでしょうか。答えは、\uwave{毎回のループを通過する前後で(各変数の値が変わっても)常に保存される性質(\textbf{ループ不変量})}「$k\cdot n!=N!$」に注目する、というアイデアです。そうすることで「階乗を出力する」という仕様が必ず満たされることを保証できます。つまり、事後条件を保証するためには、ループ不変量を探せばよいのです。これが\textbf{ホーア論理}の考え方です。\vspace{5pt}

では今度は、この事後条件「$k=N!$」を成立させるためには\uwave{事前条件として最低限どんな条件を課す必要があるか}、という問題を考えてみましょう。答えは、「$N\geq 0$」です。これを求めるには、\uwave{事後条件から出発してプログラムを後ろから前へ} \uwave{たどり、各計算に応じて条件を置き換えていく}、という操作をする必要があります。ただし、ループが何回実行されるかわからないため、この操作について不動点をとります。これが\textbf{述語変換子}、とりわけ\textbf{最弱事前条件}の考え方です。

\subsection{ホーア論理}

ホーア論理では $\{P\}\,C\,\{Q\}$ という三つ組を使います。これを\textbf{ホーア三つ組} (Hoare triple) と呼びます。これは、「事前条件$P$から始めて、プログラム$C$を実行した結果、事後条件$Q$が成り立つ」という関係です。

\begin{definition}
	ホーア論理の導出規則は以下のものからなる。
	\begin{center}	
		\AXC{} \RL{skip}
		\UIC{$\{P\}$ skip $\{P\}$}
		\DisplayProof
		\vspace{12pt}\\
		%%
		\AXC{} \RL{assignment}
		\UIC{$\{P\,[a/x]\}$ $x:=a$ $\{P\}$}
		\DisplayProof
		\vspace{12pt}\\
		%%%
		\AXC{$\{P\}\,C_1\,\{S\}$}
		\AXC{$\{S\}\,C_2\,\{Q\}$} \RL{sequential composition}
		\BIC{$\{P\}\,C_1;C_2\,\{Q\}$}
		\DisplayProof
		\vspace{12pt}\\
		%%% 
		\AXC{$\{P\land \varphi\}\,C_1\,\{Q\}$}
		\AXC{$\{P\land\lnot \varphi\}\,C_2\,\{Q\}$} \RL{if}
		\BIC{$\{P\}$ if $\varphi$ then $C_1$ else $C_2$ $\{Q\}$}
		\DisplayProof
		\vspace{12pt}\\
		%%%
		\AXC{$\{P\land \varphi\} \,C\, \{P\}$} \RL{while}
		\UIC{$\{P\}$ while $\varphi$ do $C$ $\{P\land\lnot\varphi\}$}
		\DisplayProof
		\vspace{12pt}\\
		%%% 
		\AXC{$P\Rightarrow P'$}
		\AXC{$\{P'\}\,C\,\{Q'\}$}
		\AXC{$Q'\Rightarrow Q$}
		\RL{consequence}
		\TIC{$\{P\}\,C\,\{Q\}$}
		\DisplayProof
	\end{center}
\end{definition}

\clearpage

\begin{example}
	たとえば以下は、ホーア論理の証明図の例である。\vspace{5pt}
	\begin{prooftree}
		\AXC{$\substack{x\geq 0 \land x > 0\\\Rightarrow\; x-1\geq 0}$}
		\AXC{}\RL{a}
		\UIC{$\substack{\{x-1\geq 0\}\; x:=x-1 \; \{x\geq 0\} }$} \RL{c}
		\BIC{$\substack{ \{ x\geq 0 \land x > 0\} \; x:=x-1 \; \{x\geq 0\} }$}
		%%%
		\AXC{$\substack{x\geq 0\land \lnot(x>0)\\\Rightarrow\; x\geq 0}$} 
		\AXC{}\RL{a}
		\UIC{$\substack{ \{x\geq 0\}\; x:=x\;\{x\geq 0\} }$} \RL{c}
		\BIC{$\substack{ \{x\geq 0\land\lnot(x>0)\}\; x:=x\; \{x\geq 0\} }$} \RL{i}
		\BIC{$\substack{ \{x\geq 0\} \; \textrm{if } x>0 \textrm{ then } x:=x-1  \textrm { else } x:=x \; \{x\geq 0\} }$}
	\end{prooftree}
\end{example}

% % \begin{example}
% % 	たとえば、冒頭の「階乗を計算するプログラム」のホーア論理による証明図は以下の通り。
% 	\begin{prooftree}
% 		\AXC{$\substack{N\geq 0\\\Rightarrow k\cdot n!=N!}$}
% 		% \AXC{$\substack{\{k\cdot n!=N!\land n>0\}\, k:=k\cdot n\,;\,n:=n-1\,\{k\cdot n!=N!\}}$}
% 		\AXC{$\substack{\{k\cdot n!=N!\,\land\,n>0\}\,k:=k\cdot n\,;\,n:=n-1\,\{k\cdot n!=N!\}}$}
% 		\RL{w}
% 		\UIC{$\substack{\{k\cdot n!=N!\}\,\textrm{while}\,(n>0)\,k:=k\cdot n\,;\,n:=n-1\,\{k\cdot n!=N!\,\land\,n=0\}}$}
% 		\AXC{$\substack{k\cdot n!=N!\,\land\, n=0\\\Rightarrow k=N!}$}
% 		\RL{c}
% 		\TIC{$\substack{\{N\geq 0\}\,\textrm{while}\,(n>0)\, k:=k\cdot n\,;\,n:=n-1\,\{k=N!\}}$}
% 	\end{prooftree}
% 	\begin{prooftree}
% 		\AXC{$\substack{\{N\geq 0\}\,n:=N\,\{n\geq 0\,\land\,n!=N!\}}$}
% 		\AXC{$\substack{\{n\geq 0\,\land\,n!=N!\}\,k:=1\,\{n\geq 0\,\land\,k\cdot n!=N!\}}$}
% 		\AXC{}
% 		\UIC{$\substack{\{n\geq 0\,\land\,k\cdot n!=N!\}\,\textrm{while}\,(n>0)\, k:=k\cdot n\,;\,n:=n-1\,\{k\cdot n!=N!\land n=0\}}$}
% 		\AXC{$\substack{k\cdot n!=N!\,\land\, n=0\\\Rightarrow k=N!}$}
% 		\BIC{$\substack{\{n\geq 0\,\land\,k\cdot n!=N!\}\,\textrm{while}\,(n>0)\, k:=k\cdot n\,;\,n:=n-1\,\{k=N!\}}$}
% 		\TIC{$\substack{\{N\geq 0\}\,n:=N\,;\,k:=1\,;\,\textrm{while}\,(n>0)\, k:=k\cdot n\,;\,n:=n-1\,\{k=N!\}}$}
% 	\end{prooftree}
% % \end{example}

\begin{remark}
	ループ不変量とは、以下のような while 文の証明図に現れる $I$ のこと。つまり、
	% (1) ループの入り口で事前条件によって含意され($A\Rightarrow I$)、(2) 各回のループの実行時に常に保たれ($\{I\land\varphi\}\,C\,\{I\}$)、(3) ループの出口で事後条件を含意する($I\land\lnot\varphi\Rightarrow B$)
	\begin{enumerate}
		\item ループの\kenten{入口}で事前条件によって含意され($A\Rightarrow I$)、
		\item 各回のループ\kenten{実行時}に常に保たれ($\{I\land\varphi\}\,C\,\{I\}$)、
		\item ループの\kenten{出口}で事後条件を含意する($I\land\lnot\varphi\Rightarrow B$)
	\end{enumerate}
	ような論理式$I$のことである。\vspace{5pt}
	\begin{prooftree}
		\AXC{$A\Rightarrow I$}
		\AXC{$\{I\land \varphi\}\,C\,\{I\}$} \RL{while}
		\UIC{$\{I\}\,\textrm{while}\;\varphi\;\textrm{do}\;C\,\{I\land\lnot\varphi\}$}
		\AXC{$I\land \lnot \varphi\Rightarrow B$} \RL{conseq}
		\TIC{$\{A\}\,\textrm{while}\;\varphi\;\textrm{do}\;C\,\{B\}$}
	\end{prooftree}\vspace{5pt}
	先ほどの「階乗を計算するプログラム」のホーア論理の証明図を書くと(長くなるので省略するが)$I$のところに現れる論理式はまさに「$k\cdot n!=N!$」となる。これが、先ほどのプログラムの while 文の「ループ不変量」にあたる。
\end{remark}

\subsection{述語変換子}

述語変換子のひとつである\textbf{最弱事前条件} (weakest precondition) は、述語 $Q$ を別の述語 $wp\llbracket C\rrbracket (Q)$ に変換します。この述語 $wp\llbracket C\rrbracket (Q)$ は、「プログラム $C$ について、実行後に事後条件 $Q$ を成立させるような事前条件のうち、論理的含意関係に関して最も弱いもの」です。

\begin{definition}[最弱事前条件意味論]
	状態集合を $S$ とする。$S$ 上の述語の集合を $\mathrm{Pred}(S)=\{Q\mid Q:S\to\{0,1\}\}$ と定める。なお、各述語 $Q:S\to\{0,1\}$ は、$Q=\{s\in S\mid s\models Q\}$ だと考えればよい。述語どうしの順序は、論理的含意関係によって定めるものとする。そうすると、$(\mathrm{Pred}(S),\leq)$ は完備束をなす。\[P\leq Q \quad\textrm{iff}\quad P\Rightarrow Q\] プログラム $C$ の最弱事前条件 (weakest precondition) とは、述語変換子(述語を別の述語に変換する写像) \[wp\llbracket C\rrbracket:\mathrm{Pred}(S)\to\mathrm{Pred}(S)\] で単調性 $Q_1\leq Q_2 \;\Rightarrow\; wp\llbracket C\rrbracket(Q_1)\leq wp\llbracket C\rrbracket(Q_2)$ を満たすものである。以下のように帰納的に定義される。
	\begin{itemize}
		\item $wp\llbracket\textrm{skip}\rrbracket(Q)=Q$
		\item $wp\llbracket x:=a\rrbracket(Q)=Q[a/x]$
		\item $wp\llbracket C_1;C_2\rrbracket(Q)=wp\llbracket C_1\rrbracket(wp\llbracket C_2\rrbracket(Q))$
		\item $wp\llbracket\mathrm{if}\;\varphi\;\mathrm{then}\;C_1\;\mathrm{else}\;C_2\rrbracket(Q)=(\varphi\land wp\llbracket C_1\rrbracket(Q))\lor(\lnot\varphi\land wp\llbracket C_2\rrbracket(Q))$
		\item $wp\llbracket\mathrm{while}\;\varphi\;\mathrm{do}\;C\rrbracket(Q)=\mu X.\,\underbrace{((\varphi\land wp\llbracket C\rrbracket(X))\lor(\lnot\varphi\land Q))}_{\mathrm{loop~characteristic~function~} \Phi_Q(X)}$
	\end{itemize}
\end{definition}

\begin{remark}
	ここで、$\mu$ は順序 $\leq$ についての最小不動点をとる操作である。
	% $\Phi_Q:\mathrm{Pred}(S)\to\mathrm{Pred}(S)$ はスコット連続なので、クリーネの不動点定理より、$\displaystyle\mu X.\,\Phi_Q(X)=\bigvee_{n\in\mathbb N}\Phi_Q^n(\bot)$ が得られる。
	この $\Phi_Q:\mathrm{Pred}(S)\to\mathrm{Pred}(S)$ は単調 (monotone) なので、ナスター・タルスキの定理 (the Knaster-Tarski theorem) より、最小不動点が存在する。
\end{remark}

\clearpage

% \noindent 
上記では、「ホーア三つ組」と「最弱事前条件」の2つが導入されました。ここで、両者を整理しておきましょう。
\[\begin{cases}\{P\}\,C\,\{Q\} & \textrm{Hoare-style}\\P\Rightarrow wp\llbracket C\rrbracket(Q) & \textrm{Dijkstra-style}\end{cases}\]
両者の関係として、以下が同値になることが知られています。
\[\{P\}\,C\,\{Q\} \quad\;\textrm{iff}\;\quad P\Rightarrow wp\llbracket C\rrbracket(Q)\] 
両者には、以下の違いがあります。
\begin{itemize}
	\item ホーア三つ組は「関係的」(relational)。 \\
	\quad 各述語 $P$ に対し、$\{P\}\,C\,\{Q\}$ が成り立つような述語 $Q$ は複数ありうる。\\
	\quad 各述語 $Q$ に対し、$\{P\}\,C\,\{Q\}$ が成り立つような述語 $P$ は複数ありうる。
	% \begin{itemize}
	% 	\item 各述語 $P$ に対し、$\{P\}\,C\,\{Q\}$ が成り立つような述語 $Q$ は複数ありうる。
	% 	\item 各述語 $Q$ に対し、$\{P\}\,C\,\{Q\}$ が成り立つような述語 $P$ は複数ありうる。
	% \end{itemize}
	\item 最弱事前条件は「関数的」(functional) \\
	\quad 各述語 $Q$ に対し、述語 $wp\llbracket C\rrbracket(Q)$ は一意に定まる。
	% \begin{itemize}
	% 	\item 各述語 $Q$ に対し、述語 $wp\llbracket C\rrbracket(Q)$ は一意に定まる。
	% \end{itemize}
\end{itemize}
また、wp から Hoare triple を「再現」することができます。
\begin{itemize}
	\item $\{\,wp\llbracket C\rrbracket(Q)\,\}\,C\,\{Q\}$ はいつでも成り立つ。
\end{itemize}

さて、これら Hoare-style と Dijkstra-style には、それだけでなく、より本質的な違いがあります。それは、部分正当化 (partial correctness) か、全正当性 (total correctness) か、の違いです。次節ではその点について考えます。

\subsection{部分正当性と全正当性}

プログラムの正しさを検証するうえで、\textbf{部分正当性} (partial correctness) と\textbf{全正当性} (total correctness) という2種類の「正しさ」の考え方があります。ホーア三つ組 $\{P\}\,C\,\{Q\}$ は前者(部分正当性)を主張するものです。ここでさしあたり、後者(全正当性)のほうを $[P]\,C\,[Q]$ と表記して区別することにし、両者の違いを比べてみましょう。
\begin{framed}
\begin{itemize}
	\item 部分正当性 (partial correctness) \;$\{P\}\,C\,\{Q\}$
	\begin{itemize}
		\item 「事前条件 $P$ から始めて、\kenten{もし}プログラム $C$ が\kenten{停止したら}、その実行結果は事後条件 $Q$ を満たす。」
		\item プログラムが停止しないときには何も保証しない。
	\end{itemize}
	\item 全正当性 (total correctness) \;$[P]\,C\,[Q]$
	\begin{itemize}
		\item 「事前条件 $P$ から始めて、プログラム $C$ が\kenten{必ず停止し}、\kenten{かつ}、その実行結果は事後条件 $Q$ を満たす。」
		\item プログラムの停止性 (termination) も保証する。
	\end{itemize}
\end{itemize}
\end{framed}
ホーア三つ組が保証するのは部分正当性 (partial correctness) なのに対し、最弱事前条件が保証するのは全正当性 (total correctness) です。この違いは、while ループの不動点のとり方に端を発しています。%\vspace{5pt}\\
\begin{itemize}
\item 述語変換子における while 文の解釈は、最小不動点で与えられていたことを、思い出しましょう。
\[wp\llbracket\mathrm{while}\;\varphi\;\mathrm{do}\;C\rrbracket(Q)=\mu X.\,\underbrace{((\varphi\land wp\llbracket C\rrbracket(X))\lor(\lnot\varphi\land Q))}_{\mathrm{loop~characteristic~function~} \Phi_Q(X)}\]
\item それに対し、ホーア論理における while 文の解釈は、実は、最大不動点によって与えられていたのです。\vspace{3pt}
	\begin{prooftree}
		\AXC{$P\Rightarrow I$}
		\AXC{$\{I\land \varphi\}\,C\,\{I\}$} \RL{while}
		\UIC{$\{I\}\,\textrm{while}\;\varphi\;\textrm{do}\;C\,\{I\land\lnot\varphi\}$}
		\AXC{$I\land \lnot \varphi\Rightarrow Q$} \RL{conseq}
		\TIC{$\{P\}\,\textrm{while}\;\varphi\;\textrm{do}\;C\,\{Q\}$}
	\end{prooftree}\vspace{5pt}
\end{itemize}
\clearpage

\begin{proposition}
	$L$ を完備束とし、$f:L\to L$ を単調写像とする。このとき、以下の帰結関係(上から下)が成り立つ。\vspace{5pt}
	\begin{center}
	\begin{prooftree}
		\AXC{$p\leq i$} 
		\AXC{$i\leq f(i)$} 
		\AXC{$i\leq q$} 
		\TIC{$p\leq\nu(f(-)\land q)$}
	\end{prooftree}
	\end{center}
\end{proposition}

\begin{proof}
	$\land$ の普遍性より、以下の同値関係が成り立つ。
	\begin{prooftree}
		\AXC{$i\leq f(i)$}
		\AXC{$i\leq q$}
		\doubleLine
		\BIC{$i\leq f(i)\land q$}
	\end{prooftree}
	ナスター・タルスキの定理より、\[\nu(f(-)\land q)=\bigvee\{x\in L\mid x\leq f(x)\land q\}\] であるから、特に任意の $i\in L$ について以下の帰結関係が成り立つ。
	\begin{prooftree}
		\AXC{$i\leq f(i)\land q$}
		\UIC{$i\leq\nu(f(-)\land q)$}
	\end{prooftree}
	ここで、条件 $p\leq i$ と推移性から、最終的に
	\begin{prooftree}
		\AXC{$p\leq i$} 
		\AXC{$i\leq f(i)$} 
		\AXC{$i\leq q$} 
		\TIC{$p\leq\nu(f(-)\land q)$}
	\end{prooftree}
	が得られた。
\end{proof}

\noindent 
上記の命題において、$p$ は事前条件、$q$ は事後条件、$i$ はループ不変量に、それぞれ相当しています。
% ここで、
% \begin{itemize}
% 	\item \fbox{\,$f(i)$\,} を \fbox{\,$\varphi\Rightarrow wp\llbracket C\rrbracket(I)$\,} に読み替える
% 	\item \fbox{\,$q$\,} を \fbox{\,$\lnot\varphi\Rightarrow Q$\,} に読み替える
% \end{itemize}
% とすれば、先ほどの while ループの規則に対応していることがわかりやすいのではないかと思います。
\vspace{12pt}
\begin{center}
	% \begin{prooftree}
		\AXC{$p\leq i$} 
		\AXC{$i\leq f(i)$} 
		\AXC{$i\leq q$} 
		\TIC{$p\leq\nu(f(-)\land q)$}
		\DisplayProof
	% \end{prooftree}
	\qquad$\leadsto$\qquad
	% \begin{prooftree}
		\AXC{$P\Rightarrow I$}
		\AXC{$\{I\land \varphi\}\,C\,\{I\}$} %\RL{while}
		\UIC{$\{I\}\,\textrm{while}\;\varphi\;\textrm{do}\;C\,\{I\land\lnot\varphi\}$}
		\AXC{$I\land \lnot \varphi\Rightarrow Q$} %\RL{conseq}
		\TIC{$\{P\}\,\textrm{while}\;\varphi\;\textrm{do}\;C\,\{Q\}$}
	% \end{prooftree}\vspace{5pt}
		\DisplayProof
\end{center}\vspace{12pt}
% あえて述語変換子の言葉に寄せると、以下のように書くこともできます。
% \[P\Rightarrow\nu X.\,((\varphi\Rightarrow wp\llbracket C\rrbracket(X))\land(\lnot\varphi\Rightarrow Q))\]
ホーア論理では最大不動点 (gfp) $\nu$ を、最弱事前条件では最小不動点 (lfp) $\mu$ をとっていたことがわかります。

なお、述語変換子の中にも、部分正当化をおこなうものもあり、それは「最弱リベラル事前条件」(weakest liberal precondition, wlp) と呼ばれます。これは、while ループの解釈に lfp ではなく gfp を用いる点だけが異なります。

\begin{itemize}
	\item $wp\llbracket\mathrm{while}\;\varphi\;\mathrm{do}\;C\rrbracket(Q)=\mu X.\,((\varphi\land wp\llbracket C\rrbracket(X))\lor(\lnot\varphi\land Q))$
	\item $wlp\llbracket\mathrm{while}\;\varphi\;\mathrm{do}\;C\rrbracket(Q)=\nu X.\,((\varphi\land wlp\llbracket C\rrbracket(X))\lor(\lnot\varphi\land Q))$
\end{itemize}

以上を表に整理すると、以下の通りです。\vspace{10pt}

\begin{center}
	\begin{tabular}{c|c|c}
		\hline
		& Hoare logic & predicate transformer \\ 
		% & (ホーア論理) & (述語変換子) \\
		\hline
		gfp & partial correctness & weakest liberal precondition \\
		(最大不動点) & $\{P\}\,C\,\{Q\}$ & $wlp\llbracket C\rrbracket(Q)$ \\[3pt]
		& \!「$P$から始めて、\kenten{もし}停止する\kenten{なら}$Q$」\! & 「\kenten{もし}停止する\kenten{なら}$Q$を満たす最弱条件」 \\
		\hline
		lfp & total correctness & weakest precondition \\ 
		(最小不動点) & $[P]\,C\,[Q]$ & $wp\llbracket C\rrbracket(Q)$\\[3pt]
		& \!「$P$から始めて、\kenten{必ず}停止して\kenten{かつ}$Q$」\! & 「\kenten{必ず}停止して\kenten{かつ}$Q$を満たす最弱条件」 \\
		\hline
	\end{tabular}
\end{center}
\vspace{10pt}
直観的には、一般に順序集合における最小不動点と最大不動点の間には $\mu x.\,Fx\leq \nu x.\, Fx$ という大小関係があるわけですが、このギャップが「プログラムの停止性を保証するかしないか」の違いだ、と理解するのがわかりやすいでしょう。
\vspace{10pt}

% \clearpage

% % \subsection{Almost Sure Termination ならば lfp=gfp}
% \subsection{while ループが停止するとき lfp\,=\,gfp(不動点の一致)}

% あらためて「部分正当性」と「全正当性」の違いを確認しておきましょう。

% \begin{framed}
% 	\noindent 復習:
% 	\begin{itemize}
% 	\item 部分正当性 (partial correctness) \;$\{P\}\,C\,\{Q\}$
% 	\begin{itemize}
% 		\item 「事前条件 $P$ から始めて、\kenten{もし}プログラム $C$ が\kenten{停止したら}、その実行結果は事後条件 $Q$ を満たす。」
% 		\item プログラムが停止しないときには何も保証しない。
% 	\end{itemize}
% 	\item 全正当性 (total correctness) \;$[P]\,C\,[Q]$
% 	\begin{itemize}
% 		\item 「事前条件 $P$ から始めて、プログラム $C$ が\kenten{必ず停止し}、\kenten{かつ}、その実行結果は事後条件 $Q$ を満たす。」
% 		\item プログラムの停止性 (termination) も保証する。
% 	\end{itemize}
% \end{itemize}
% \end{framed}

% ここで、プログラム $C$ が停止することがわかっている場合、

% ……

% \subsection{次章へ向けて}

さて、ここまで抽象的な話が続いたので、次節でいったん具体例を見ておきましょう。

\clearpage

\subsection{述語変換子を用いたモデル検査の簡単な例}

本節では、述語変換子(前節で導入した最弱事前条件に限らず一般の述語変換子)を使って実際にモデル検査をする方法を、簡単な例を挙げて紹介します。以下では
例として、クリプキフレーム $(X,\delta:X\to\mathcal P(X))$ を考えます。$X$ は状態集合で、$\delta$ は遷移関数です。(これは次章で導入する用語で言うと「余代数」なのですが、詳しくは後で述べます。)
\[\begin{array}{cccl}\delta:&X&\longrightarrow&\mathcal P(X)\\&\rotatebox{90}{$\in$}&&~~\rotatebox{90}{$\in$}\\&x&\longmapsto&\delta(x)=\{x'\mid x\to x'\}\end{array}\]
絵で描くとたとえば次のようなものです。
\begin{center}
	\begin{tikzcd}[row sep=2pt]
		& \circled{$x_1$} \\[-1pt]
		\circled{$x_0$} \ar[ur] \ar[r] \ar[dr] & \circled{$x_2$} \\[2pt]
		& \cdots 
	\end{tikzcd}
\end{center}
$L$を完備束とし、以下で定められているものとします。
\[L:=2^X=\{0,1\}^X\]
この$L$は述語の集合であり(補足: $p(x)=1$ を $x\models p$ と考えます)、$L$上の順序$\leq$は述語どうしの論理的含意関係によって定められている、と考えます。なお、$L$の最小元と最大元をそれぞれ $\bot,\top$ と書くことにします。

ここで、述語変換子 $f:2^X\to 2^X$ を、次のように定めます。(補足: この $f$ は単調 (monotone) な関数です。)
\[\begin{array}{cccl}f:&2^X&\longrightarrow&2^X\\&\rotatebox{90}{$\in$}&&\rotatebox{90}{$\in$}\\&p&\longmapsto & \{x\mid \forall x' ~ (x'\in\delta(x)\Rightarrow x'\in p)\}\\[3pt] &&&=\{x\mid \textrm{all successors of $x$ satisfy $p$}\}.\end{array}\]
そして、この $f$ の最小不動点 $\mu f$ をとります。(補足: この$\mu f$は、$L$の要素、すなわち述語です。)
\[\mu f=\{x\mid\textrm{every path from $x$ reaches a state with no successors}\}\]
$\mu f$ は、以下の図に見られるような$2^X$における鎖の、超限反復 (transfinite iteration) によって構成されます。\\(最小不動点 $\mu f$ の近似列は、下から積み上げます。)
\begin{center}
	\begin{small}
	\begin{tikzcd}[row sep=0pt]
		\bot \ar[r,"\leq"] \ar[rrrr,bend left=60pt] & f(\bot) \ar[r,"\leq"] \ar[rrr,bend left=50pt] & f^2(\bot) \ar[r,"\leq"] \ar[rr,bend left=40pt] & ~~\cdots~~ \ar[r,"\leq"] & f^\omega (\bot) \ar[r,"\leq"] & f^{\omega+1}(\bot) \ar[r,"\leq"] & ~~\cdots~~ \\ % & \textrm{(in $2^X$)}\\
		\rotatebox{90}{$=$} & \rotatebox{90}{$=$} & \rotatebox{90}{$=$} && \rotatebox{90}{$=$}  \\
		\emptyset & \{x\mid\delta(x)=\emptyset\} & \!\!\substack{\displaystyle\{x\mid \textrm{all paths are}\\\displaystyle~~~~~~\textrm{of length $\leq 2$}\}}\!\! && ~\underset{\beta<\omega}{\sup}\,f^\beta(\bot)~
	\end{tikzcd}
	\end{small}
\end{center}\vspace{15pt}

\noindent (補足: 一般に、$f$は単調ですが、スコット連続であるとは限らないため、$\omega$ よりも先まで(安定 (stabilize) するまで)反復を回し続ける必要があります。)

\clearpage

ここで具体的に、以下のクリプキフレーム $(X,\delta)$ について、上で定めた述語 $\mu f$ を実際に検証してみましょう。\vspace{8pt}
\begin{center}
\begin{small}
\begin{tikzcd}[column sep=2pt,row sep=15pt]
	&&\circled{\hspace*{2pt}$x_\varepsilon$\hspace*{2pt}} \ar[lld,dash] \ar[ld,dash] \ar[d,dash] \ar[rd,dash] \ar[rrd,dash] \ar[rrrd,dash] \\
	\doublecircled{$x_{00}$} & \circled{$x_{10}$} \ar[d,dash] & \circled{$x_{20}$} \ar[d,dash] & \cdots & \circled{$x_{n0}$} \ar[d,dash] & ~~~\cdots~~~ \\
	& \doublecircled{$x_{11}$} & \circled{$x_{21}$} \ar[d,dash] & & \circled{$x_{n1}$} \ar[d,dash,shorten >= -4pt] & \\
	&& \doublecircled{$x_{22}$} && \vdots \ar[d,dash,shorten <= 2pt] & \\
	&&&& \doublecircled{$x_{nn}$}
\end{tikzcd}
\end{small}
\end{center}\vspace{8pt}
$x_\varepsilon\models\mu f$ が成り立つことが、以下の図のような超限反復からわかります。(この鎖は、$\omega+1$で安定 (stabilize) し、そこから先は変わらない、つまり最小不動点となります。)

\begin{framed}
\noindent
\begin{tiny}
\begin{tikzcd}[column sep=2pt,row sep=15pt]
	&& \scalebox{1.5}{\!$\bot$\!} &&& \\
	&&\circled{$0$} \ar[lld,dash] \ar[ld,dash] \ar[d,dash] \ar[rd,dash] \ar[rrd,dash] \ar[rrrd,dash] \\
	\doublecircled{$0$} & \circled{$0$} \ar[d,dash] & \circled{$0$} \ar[d,dash] & \cdots & \circled{$0$} \ar[d,dash] & ~~~\cdots~~~ \\
	& \doublecircled{$0$} & \circled{$0$} \ar[d,dash] & & \circled{$0$} \ar[d,dash,shorten >= -4pt] & \\
	&& \doublecircled{$0$} && \vdots \ar[d,dash,shorten <= 2pt] & \\
	&&&& \doublecircled{$0$}
\end{tikzcd}
$\xrightarrow{~~\leq~~}$
\begin{tikzcd}[column sep=2pt,row sep=15pt]
	&& \scalebox{1.5}{\!\!$f(\bot)$\!\!} &&& \\
	&&\circled{$0$} \ar[lld,dash] \ar[ld,dash] \ar[d,dash] \ar[rd,dash] \ar[rrd,dash] \ar[rrrd,dash] \\
	\doublecircled{$1$} & \circled{$0$} \ar[d,dash] & \circled{$0$} \ar[d,dash] & \cdots & \circled{$0$} \ar[d,dash] & ~~~\cdots~~~ \\
	& \doublecircled{$1$} & \circled{$0$} \ar[d,dash] & & \circled{$0$} \ar[d,dash,shorten >= -4pt] & \\
	&& \doublecircled{$1$} && \vdots \ar[d,dash,shorten <= 2pt] & \\
	&&&& \doublecircled{$1$}
\end{tikzcd} 
%\vspace{15pt}\\ %
$\xrightarrow{~~\leq~~}$
\begin{tikzcd}[column sep=2pt,row sep=15pt]
	&& \scalebox{1.5}{\!\!$f^2(\bot)$\!\!} &&& \\
	&&\circled{$0$} \ar[lld,dash] \ar[ld,dash] \ar[d,dash] \ar[rd,dash] \ar[rrd,dash] \ar[rrrd,dash] \\
	\doublecircled{$1$} & \circled{$1$} \ar[d,dash] & \circled{$0$} \ar[d,dash] & \cdots & \circled{$0$} \ar[d,dash] & ~~~\cdots~~~ \\
	& \doublecircled{$1$} & \circled{$1$} \ar[d,dash] & & \circled{$0$} \ar[d,dash,shorten >= -4pt] & \\
	&& \doublecircled{$1$} && \vdots \ar[d,dash,shorten <= 2pt] & \\
	&&&& \doublecircled{$1$}
\end{tikzcd}
$\xrightarrow{~~\leq~~}\;\cdots\;$
\vspace{20pt}\\ %
$\xrightarrow{~~\leq~~}$
\begin{tikzcd}[column sep=2pt,row sep=15pt]
	&& \scalebox{1.5}{\!\!$f^n(\bot)$\!\!} &&& \\
	&&\circled{$0$} \ar[lld,dash] \ar[ld,dash] \ar[d,dash] \ar[rd,dash] \ar[rrd,dash] \ar[rrrd,dash] \\
	\doublecircled{$1$} & \circled{$1$} \ar[d,dash] & \circled{$1$} \ar[d,dash] & \cdots & \circled{$1$} \ar[d,dash] & ~~~\cdots~~~ \\
	& \doublecircled{$1$} & \circled{$1$} \ar[d,dash] & & \circled{$1$} \ar[d,dash,shorten >= -4pt] & \\
	&& \doublecircled{$1$} && \vdots \ar[d,dash,shorten <= 2pt] & \\
	&&&& \doublecircled{$1$}
\end{tikzcd}
$\xrightarrow{~~\leq~~}\;\cdots\;\xrightarrow{~~\leq~~}$
\begin{tikzcd}[column sep=2pt,row sep=15pt]
	&& \scalebox{1.5}{\!\!$f^\omega(\bot)$\!\!} &&& \\
	&&\circled{$0$} \ar[lld,dash] \ar[ld,dash] \ar[d,dash] \ar[rd,dash] \ar[rrd,dash] \ar[rrrd,dash] \\
	\doublecircled{$1$} & \circled{$1$} \ar[d,dash] & \circled{$1$} \ar[d,dash] & \cdots & \circled{$1$} \ar[d,dash] & ~~~\cdots~~~ \\
	& \doublecircled{$1$} & \circled{$1$} \ar[d,dash] & & \circled{$1$} \ar[d,dash,shorten >= -4pt] & \\
	&& \doublecircled{$1$} && \vdots \ar[d,dash,shorten <= 2pt] & \\
	&&&& \doublecircled{$1$}
\end{tikzcd}
% \vspace{15pt}\\ %
$\xrightarrow{~~\leq~~}$
\begin{tikzcd}[column sep=2pt,row sep=15pt]
	&& \scalebox{1.5}{\!\!\!\!\!\!$f^{\omega+1}(\bot)$\!\!\!\!\!\!} &&& \\
	&&\circled{$1$} \ar[lld,dash] \ar[ld,dash] \ar[d,dash] \ar[rd,dash] \ar[rrd,dash] \ar[rrrd,dash] \\
	\doublecircled{$1$} & \circled{$1$} \ar[d,dash] & \circled{$1$} \ar[d,dash] & \cdots & \circled{$1$} \ar[d,dash] & ~~~\cdots~~~ \\
	& \doublecircled{$1$} & \circled{$1$} \ar[d,dash] & & \circled{$1$} \ar[d,dash,shorten >= -4pt] & \\
	&& \doublecircled{$1$} && \vdots \ar[d,dash,shorten <= 2pt] & \\
	&&&& \doublecircled{$1$}
\end{tikzcd}
$\xrightarrow{~~\leq~~}\;\cdots\;$
\end{tiny}
\end{framed}
% \vspace{10pt}

\noindent 補足: $f^\omega(\bot)$ は、これより左に現れるすべての上限 (supremum) をとったものです。(てっぺん $x_\varepsilon$ にはまだ $1$ が現れたことがないので、上限は $0$ のままです。)それに対し、次の $f^{\omega+1}(\bot)$ でやっと、てっぺん $x_\varepsilon$ にも $1$ が現れます。

\clearpage

$\omega+1$よりもさらも先で安定 (stabilize) する例もあります。分岐がもっと複雑なものを考えればよいです。たとえば、以下のようなクリプキフレームを考えてみましょう。この場合、$\omega+2$で安定します。
\vspace{8pt}
%
\begin{multicols}{2} %%%
	\begin{center}
	\begin{small}
	\begin{tikzcd}[column sep=2pt,row sep=15pt]
		&&\circled{\hspace*{2pt}$x_{\varepsilon}$\hspace*{2pt}} \ar[d,dash] \\
		&&\circled{\hspace*{4pt}$y$\hspace*{4pt}} \ar[lld,dash] \ar[ld,dash] \ar[d,dash] \ar[rd,dash] \ar[rrd,dash] \ar[rrrd,dash] \\
		\doublecircled{$x_{00}$} & \circled{$x_{10}$} \ar[d,dash] & \circled{$x_{20}$} \ar[d,dash] & \cdots & \circled{$x_{n0}$} \ar[d,dash] & ~~~\cdots~~~ \\
		& \doublecircled{$x_{11}$} & \circled{$x_{21}$} \ar[d,dash] & & \circled{$x_{n1}$} \ar[d,dash,shorten >= -4pt] & \\
		&& \doublecircled{$x_{22}$} && \vdots \ar[d,dash,shorten <= 2pt] & \\
		&&&& \doublecircled{$x_{nn}$}
	\end{tikzcd}
	\end{small}
	\end{center}
	% 
	~\\[25pt]
	(以降、左の図を以下のように略記することにします。)\\[10pt]
	\begin{center}
		\begin{small}
		\begin{tikzcd}[column sep=0pt]
			& \circled{\hspace*{2pt}$x_\varepsilon$\hspace*{2pt}} \ar[d,dash] & \\
			& \circled{\hspace*{4pt}$y$\hspace*{4pt}} \ar[ld,dash,shorten <= -5pt,shorten >= -2.2pt] \ar[rd,dash,shorten <= -5pt,shorten >= -2.2pt] & {} \\[8pt]
			{} \ar[rr,dash,shorten = -4.8pt] & {} & {}
		\end{tikzcd}
		\end{small}
	\end{center}
\end{multicols}
\vspace{8pt}

% さらに、上の図を以下のように略記することにし、
% \vspace{8pt}
% \begin{center}
% \begin{small}
% \begin{tikzcd}[column sep=0pt]
% 	& \circled{\hspace*{2pt}$x_\varepsilon$\hspace*{2pt}} \ar[d,dash] & \\
% 	& \circled{\hspace*{4pt}$y$\hspace*{4pt}} \ar[ld,dash,shorten <= -5pt,shorten >= -2.2pt] \ar[rd,dash,shorten <= -5pt,shorten >= -2.2pt] & {} \\
% 	{} \ar[rr,dash,shorten = -4.8pt] & {} & {}
% \end{tikzcd}
% \end{small}
% \end{center}\vspace{8pt}
さらに、(上記の略記を用いて)もっと複雑なものを考えてみます。たとえば、以下の図のクリプキフレームを考えましょう。この場合、$\omega\cdot 2+1$ で安定します。
\vspace{8pt}
\begin{center}
\begin{small}
% \begin{tikzcd}[column sep=0pt,row sep=10pt]
% 	&&&&&& \circled{\hspace*{2pt}$x_\varepsilon$\hspace*{2pt}} \ar[llllld,dash] \ar[lld,dash] \ar[rrd,dash] &&&&\\[8pt]
% 	& \circled{$y_{00}$} \ar[ld,dash,shorten <= -5pt,shorten >= -2.2pt] \ar[rd,dash,shorten <= -5pt,shorten >= -2.2pt] & & & \circled{$y_{10}$} \ar[d,dash] & & \cdots & & \circled{$y_{n0}$} \ar[d,dash] & &\cdots \\
% 	{} \ar[rr,dash,shorten = -4.8pt] & {} & {} & & \circled{$y_{11}$} \ar[ld,dash,shorten <= -5pt,shorten >= -2.2pt] \ar[rd,dash,shorten <= -5pt,shorten >= -2.2pt]  & & & & \circled{$y_{1n}$} \ar[d,dash,shorten >= -4pt] & \\
% 	&&& {} \ar[rr,dash,shorten = -4.8pt] & {} & {} & & & \vdots \ar[d,dash,shorten >= 2pt] & \\
% 	&&&&&&&& \circled{$y_{nn}$} \ar[ld,dash,shorten <= -5pt,shorten >= -2.2pt] \ar[rd,dash,shorten <= -5pt,shorten >= -2.2pt] & \\
% 	&&&&&&& {} \ar[rr,dash,shorten = -4.8pt] & {} & {}
% \end{tikzcd}
\begin{tikzcd}[column sep=0pt,row sep=12pt]
	&&&&&&& \circled{\hspace*{2pt}$x_\varepsilon$\hspace*{2pt}} \ar[lllllld,dash] \ar[llld,dash] \ar[d,dash] \ar[rrd,dash] \ar[rrrrd,dash] \ar[rrrrrrd,dash] &&&&&&\\[15pt]
	& \circled{$y_{00}$} \ar[ld,dash,shorten <= -5pt,shorten >= -2.2pt] \ar[rd,dash,shorten <= -5pt,shorten >= -2.2pt] &&& \circled{$y_{10}$} \ar[d,dash] &&& \circled{$y_{20}$} \ar[d,dash] && \cdots && \circled{$y_{n0}$} \ar[d,dash] && \cdots \\
	{} \ar[rr,dash,shorten = -4.8pt] & {} & {} && \circled{$y_{11}$} \ar[ld,dash,shorten <= -5pt,shorten >= -2.2pt] \ar[rd,dash,shorten <= -5pt,shorten >= -2.2pt] &&& \circled{$y_{21}$} \ar[d,dash] &&&& \circled{$y_{n1}$} \ar[d,dash] & \\
	&&& {} \ar[rr,dash,shorten = -4.8pt] & {} & {} && \circled{$y_{22}$} \ar[ld,dash,shorten <= -5pt,shorten >= -2.2pt] \ar[rd,dash,shorten <= -5pt,shorten >= -2.2pt]  &&&& \circled{$y_{n2}$} \ar[d,dash,shorten >= -4pt] & \\
	&&&&&&{} \ar[rr,dash,shorten = -4.8pt] &{}&{}&&&\vdots \ar[d,dash,shorten >= 2pt] & \\
	&&&&&&&&&&& \circled{$y_{nn}$} \ar[ld,dash,shorten <= -5pt,shorten >= -2.2pt] \ar[rd,dash,shorten <= -5pt,shorten >= -2.2pt] & \\[15pt]
	&&&&&&&&&&{} \ar[rr,dash,shorten = -4.8pt] &{}&{} 
\end{tikzcd}
\end{small}
\end{center}\vspace{35pt}

さて、本章でここまで扱ってきた述語変換子の考え方は、圏論の言葉を使うととても見通しがよくなります。そのための準備として、次章では、圏論の関手を使ってプログラムの振舞いを捉える方法について紹介します。(すでにだいたい知っているよ、という人は、3章まで飛ばしてください。)

\clearpage

\section{圏論の関手でシステムの振舞いを捉える}

情報科学における「圏論的手法の強み」は、抽象的・俯瞰的であるがゆえに、様々な種類の情報システムに対して、既存の議論や知見を再編することで新たな問題設定に柔軟に適応できる点にあります。

本章で紹介する\textbf{余代数} (coalgebra) は、そのように多様な状態遷移システム(例: 二分木グラフ、ストリーム、クリプキフレーム、オートマトン、マルコフ連鎖、マルコフ決定過程、etc.)を圏論的に扱う際にとても役立つ道具です。

\subsection{余代数の定義}

まずは余代数の定義を述べます。

\begin{definition}[余代数]
	$\mathbb{C}$ を圏とし、$F:\mathbb{C}\to\mathbb{C}$ を関手とする。$F$-余代数とは、$\mathbb C$ の対象 $X$ と射 $c:X\to F(X)$ の組 $(X,c)$ のことと定める。また、2つの余代数 $(X,c:X\to F(X)),\,(Y,d:Y\to F(Y))$ の間の余代数準同型とは、射 $f:X\to Y$ であって $d\circ f=F(f)\circ c$ を満たすものとする。\vspace{5pt}
	\begin{center}
		\begin{tikzcd}[row sep=2pt, column sep=4pt]
			F(X) \ar[rr,"F(f)"] & & F(Y)\\
			& \rotatebox{-45}{$=$} & \\
			X \ar[uu,"c"] \ar[rr,"f"'] & & Y \ar[uu,"d"']
		\end{tikzcd}
	\end{center}
\end{definition}

\begin{definition}[余代数の圏]
	$F$-余代数の圏を $\Coalg(F)$ と書くことにする。
	% 対象が余代数であり、射が余代数準同型であるような圏である。
	対象が余代数、射が余代数準同型である。
	% \begin{small}
		\begin{align*}
			\begin{small}
			\left(\begin{array}{ccc}F(X)\\~\;\uparrow{\scriptstyle c}\\X\end{array}\right)\xrightarrow{~~f~~}\left(\begin{array}{ccc}F(Y)\\~\;\uparrow{\scriptstyle d}\\Y\end{array}\right)
			\end{small}
		\end{align*}
	% \end{small}
\end{definition}

\subsection{余代数の様々な例} \label{subsec:examples}
いくつか例を挙げます。
\begin{example}
	二分木グラフは、$\Set$ 上の関手 $F(X)=1+(X\times X)$ に対する余代数 $c:X\to F(X)$ として与えられる。
	\[\textrm{分岐}\quad c(x)=(x',x'') \qquad\textrm{or}\qquad \textrm{停止}\quad c(x)=\ast\]
	\begin{small}
	\begin{center}
		\begin{tikzcd}[row sep=15pt,column sep=6pt]
			& \circled{\hspace*{4pt}$x$\hspace*{4pt}} \ar[ld,shorten=-5pt] \ar[rd,shorten=-5pt] & \\
			 \circled{\hspace*{2pt}$x'$\hspace*{2pt}} && \circled{\hspace*{1pt}$x''$\hspace*{1pt}}
		\end{tikzcd}
		\qquad\qquad 
		\begin{tikzcd}[row sep=15pt,column sep=6pt]
			\circled{\hspace*{4pt}$x$\hspace*{4pt}} \ar[d,shorten=-2pt,"\backslash"marking]\\
			{~~}
		\end{tikzcd}
	\end{center} \vspace{2pt}
	\end{small}
	たとえば以下のような二分木グラフが挙げられる。\vspace{2pt}
	\begin{center}
		\begin{tikzcd}[row sep=8pt,column sep=12pt]
			& \circled{} \ar[ld,dash] \ar[rd,dash] \\
			\circled{} \ar[d,dash] & & \circled{} \ar[ld,dash] \ar[rd,dash] \\
			\ast & \circled{} \ar[d,dash] & {} & \circled{} \ar[ld,dash] \ar[rd,"\ddots", phantom] \\
			& \ast & \circled{} \ar[d,dash] & {} & {} \\
			&& \ast & 
		\end{tikzcd}
	\end{center}
\end{example}

% \begin{example}
% 	ストリームは、$\Set$ 上の関手 $F(X)=A\times X$ に対する余代数 $c:X\to F(X)$ として与えられる。
% 	\[c(x)=(\head(x),\tail(x))\]
% 	たとえば、ストリーム $x=(x_1,x_2,x_3,\dots)$ について、$\head(x)=x_1,\,\tail(x)=(x_2,x_3,\dots)$ である。\vspace{10pt}

% 	\begin{center}
% 		\begin{tikzcd}
% 			{} \ar[r,shorten <=16pt,shorten >= -3pt]&\underset{abbbbb\dots}{\circled{\hspace*{1pt}$q_0$\hspace*{1pt}}} \ar[r,"a"] & \underset{~~bbb\dots}{\underset{~bbbb\dots}{\underset{bbbbb\dots}{\circled{\hspace*{1pt}$q_1$\hspace*{1pt}}}}} \ar[loop right,"b"]
% 		\end{tikzcd}
% 	\end{center}
% \end{example}

\clearpage

\begin{example}
	様相論理におけるクリプキフレーム (Kripke frame) は、余代数 $\delta:X\to\mathcal P(X)$ で与えられる。\[\begin{array}{cccl}\delta:&X&\longrightarrow&\mathcal P(X)\\&\rotatebox{90}{$\in$}&&~~\rotatebox{90}{$\in$}\\&x&\longmapsto&\delta(x)=\{x'\mid x\to x'\}\end{array}\]
	\begin{small}
	\begin{center}
		\begin{tikzcd}[row sep=0pt]
			& \circled{$x_1$} \\[-1pt]
			\circled{$x_0$} \ar[ur] \ar[r] \ar[dr] & \circled{$x_2$} \\[2pt]
			& \cdots 
		\end{tikzcd}
	\end{center}
	\end{small}
\end{example}

\begin{example}%[決定的オートマトン・非決定的オートマトン]
	非決定的オートマトンは、余代数 $\gamma:X\to 2\times\mathcal P(A\times X)$ で与えられる。なお $2=\{\bigcirc,\doublecirc\}$ とする。
	\[\gamma(x)=\left(\doublecirc,\{(a,x'),\dots\}\right)\quad\textrm{or}\quad\gamma(x)=\left(\bigcirc,\{(a,x'),\dots\}\right)\]
	% 非決定的オートマトンは、余代数 $\gamma:X\to \mathcal P(A\times X)$ で与えられる。
	% \[\gamma(x)=\{(a,x'),\dots\}\]
	\begin{center} \small
		% \begin{tikzcd}[row sep=2pt,column sep=5pt, /tikz/column 2/.append style={column sep={55pt}}, /tikz/column 1/.append style={column sep={15pt}}]
		\begin{tikzcd}[row sep=5pt,column sep=35pt]
			{} \ar[r,shorten <= 15pt] & \circled{$x_1$} \ar[r,"a"] \ar[rd,"c"] & \circled{$x_2$} \ar[loop above,"b"] \ar[r,"b"] & \doublecircled{$x_3$} \\[-2pt]
			&& \circled{$x_4$}
			% \\\\\hline\\\\
			% &x_1 \ar[r,mapsto] & \gamma(x_1) \ar[d,equal] \\[4pt]
			% && \{(a,x_2),(b,x_4)\} \ar[r,mapsto] & \{(a,\gamma(x_2)),(b,\gamma(x_4))\} \ar[d,equal] \\[4pt]
			% &&& \{(a,\{(b,x_2),(c,x_3)\}),(b,\{\})\} \ar[d,mapsto,"T\lambda_{FX}"]\\[12pt]
			% &&& \{\{ (ab,x_2),(ac,x_3) \},\{\}\} \ar[d,mapsto,"\mu_{F^2X}"] \\[12pt]
			% &&& \{(ab,x_2),(ac,x_3)\}
		\end{tikzcd}
	\end{center}
	% これは、非決定的分岐を表す関手 $T(-)=\mathcal P(-)$ 、1ステップの振舞いを表す関手 $F(-)=A\times (-)$、受理かどうかを表す関手 $G(-)=2\times(-)$ を組み合わせた余代数 $\gamma:X\to G(T(F(X)))$ である。なお、図中の $\mu$ はモナドの積 (multiplication)、$\lambda$ はモナドの分配則 (distributive law) を表しているが、これらについては後述する。
	% なお、非決定的オートマトンは、余代数 $\gamma:X\to 2\times\mathcal P(X)^A$ で与えられることもある。この見方をとると「非決定的オートマトンの決定化」(べき集合構成)を扱える。すなわち、非決定的オートマトン $\gamma:X\to\mathcal P(X)^A$ 
	% 		\begin{multicols}{2}
	% 			\begin{center}\small
	% 				\begin{tikzcd}
	% 					{\!\!\!\!} \ar[r,shorten <= 2mm] & \circled{$x_1$} \ar[r,"a",shift left=8pt] \ar[r,"b"] \ar[r,leftarrow,"a",shift right=8pt] \ar[loop above,looseness=2mm,"a"] &\doublecircled{$x_2$} 
	% 				\end{tikzcd}
	% 			\end{center}
	% 			% \begin{align*}
	% 			% 	\gamma(x_1) &=\left[\begin{array}{ccl} a &\mapsto& \{x_1,x_2\}\\ b&\mapsto &\{x_1\}\end{array}\right] \\
	% 			% 	\gamma(x_2) &=\left[\begin{array}{ccl} a &\mapsto& \emptyset\\ b&\mapsto &\emptyset\end{array}\right]
	% 			% \end{align*}
	% 			\begin{align*}
	% 				\gamma(x_1) &=\left[\begin{array}{ccl} a &\mapsto& \{x_1,x_2\}\\ b&\mapsto &\{x_2\}\end{array}\right] \\
	% 				\gamma(x_2) &=\left[\begin{array}{ccl} a &\mapsto& \{x_1\}\\ b&\mapsto &\emptyset\end{array}\right]
	% 			\end{align*}
	% 		\end{multicols}
	% 		から決定的オートマトン
	% 		$\delta=(F\mu_X\circ \lambda_{TX}\circ T\gamma): \mathcal P(X)\to\mathcal P(X)^A$ 
	% 		を構成することができる。
	% 		\begin{multicols}{2}
	% 			\begin{center}\small
	% 				\begin{tikzcd}[row sep=-16pt]
	% 					{\!\!\!\!\!} \ar[r,shorten <= 2mm] & \circled{\small $\{x_1\}$}  \ar[rd,"a"] & {}
	% 					\\&& \doublecircled{\scriptsize $\{x_1,x_2\}$} \ar[loop right,looseness = 1.5mm, "a"] \ar[ld,"b"] \\
	% 					& \doublecircled{\small$\{x_2\}$} \ar[uu,"a",shift left=1mm] \ar[uu, leftarrow, "b"', shift right=1mm]
	% 				\end{tikzcd}
	% 			\end{center}
	% 			\begin{align*}
	% 				\delta(\{x_1\}) &=\left[\begin{array}{ccl} a &\mapsto& \{x_1,x_2\}\\ b&\mapsto &\{x_2\}\end{array}\right] \\
	% 				\delta(\{x_2\}) &= \left[\begin{array}{ccl} a &\mapsto& \{x_1\}\\ b&\mapsto &\emptyset\end{array}\right]
	% 				\\
	% 				\delta(\{x_1,x_2\}) &=\left[\begin{array}{ccl} a &\mapsto& \{x_1,x_2\}\\ b&\mapsto &\{x_2\}\end{array}\right]
	% 			\end{align*}
	% 		\end{multicols}
	% なお、両者の捉え方がいずれも同じものであることは直ちにわかる。
	% \begin{prooftree}
	% 	\AXC{$X\to\mathcal P(X)^{A}$} \doubleLine \RL{curry}
	% 	\UIC{$X\times A\to\mathcal P(X)$} \doubleLine \RL{classify}
	% 	\UIC{$\textrm{relation}\subseteq (X\times A)\times X$} \doubleLine \RL{associate}
	% 	\UIC{$\textrm{relation}\subseteq X\times (A\times X)$} \doubleLine \RL{classify}
	% 	\UIC{$X\to\mathcal P(A\times X)$}
	% \end{prooftree}
\end{example}

% \clearpage

\begin{example}
	マルコフ連鎖は、余代数 $\gamma:X\to \mathcal D(X)$ で与えられる。なお、関手 $\mathcal D:\Set\to\Set$ は
	\begin{align*}\mathcal D(X)&=\{d:X\to [0,1]\mid\textrm{finite support,~}\textstyle\sum d(x)=1\}\\&=\{\textstyle\,\sum p_i\ket{x_i}\,\textrm{, where each $p_i$ is a probability $d(x_i)\in[0,1]$.}\}\\\mathcal D(f)&:\mathcal D(X)\to\mathcal D(Y)\\&\mathcal D(f)(\textstyle\sum p_i\ket{x_i})=\textstyle\sum p_i\ket{f(x_i)}\end{align*}
	と定められているものとする。
	% \begin{multicols}{2}
	\begin{center}
		\begin{tikzcd}[row sep=-7pt]
			\circled{$x_1$} \ar[rd,shift left,"0.3"] \ar[dd,"0.5"'] \ar[loop above, looseness=5,"0.2"] & \\
			& \circled{$x_2$} \ar[dl,"0.7"] \ar[ul,"0.3",shift left=1mm] \\
			\circled{$x_3$} \ar[loop below,looseness=5, "1"] 
		\end{tikzcd}
	\end{center}
		\begin{align*}\gamma(x_1) & =0.2\ket{x_1}+0.3\ket{x_2}+0.5\ket{x_3},\\[-2pt]\gamma(x_2) & = 0.3\ket{x_1}+0.7\ket{x_3}, \\[-2pt] \gamma(x_3) & = \ket{x_3}. \end{align*}
	% \end{multicols}
\end{example}

\begin{remark}[分布関手のためのケット記法]
	% 分布に対して次の記法を用いる。
	$\mathcal D(X)=\{\, d: X \to [0,1]\mid \sum d(x)=1 \,\}$ の各要素を $\sum p_i\lvert x_i\rangle$ と書く。ここで各 $p_i$ は確率 $d(x_i)\in[0,1]$ である。例えば、コイントス $X=\{H,T\}$ を考えるとき、「公平なコイン」の分布は $d=\tfrac12\lvert H\rangle+\tfrac12\lvert T\rangle$ と書ける。なお $\lvert H\rangle=\left(\begin{smallmatrix}1\\0\end{smallmatrix}\right)$ および $\lvert T\rangle=\left(\begin{smallmatrix}0\\1\end{smallmatrix}\right)$ である。
\end{remark}

\clearpage

% \subsection{終余代数とその構成}

% ここでは、終余代数の定義と、その構成のしかたを紹介します。まずは定義です。

% \begin{definition}[終余代数]
	
% \end{definition}

% \begin{proposition}[ランベックの補題]
	
% \end{proposition}

% \begin{corollary}
% 	べき集合の関手 $\mathcal P:\Set\to\Set$ には、終余代数は存在しない。
% \end{corollary}

% \begin{proof}
	
% \end{proof}

% ここからは、どうやって終余代数を構成するかについて見ていきます。
% % 終余代数の構成は、順序集合についての「クーゾー・クーゾーの定理」(the Cousot-Cousot theorem) を一般化したものなので、まずはその復習から始めましょう。
% % \begin{proposition}[クーゾー・クーゾーの定理]
% % 	% 
% % \end{proposition}




\clearpage

\subsection{計算的作用 (computational effect) について}

\ref{subsec:examples}節でいくつか例を挙げながら紹介したように、余代数 (coalgebra) は「1ステップの状態遷移」を表します。\[c:X\to F(X)\] 

\noindent より一般に、計算 (computation) は、以下のように表されます。(なお $X=Y$ とすれば余代数です。)\[f:X\to F(Y)\] 

\noindent ここで「ただの関数」(pure function) と「作用つき計算」(effectful computation) を区別することは重要です。%\[\textrm{pure function} \quad f:X\to Y \qquad \textrm{vs} \qquad \textrm{effectful computation} \quad c:X\to F(Y)\]
\begin{framed}
\[\textrm{pure function} \qquad \textrm{vs} \qquad \textrm{effectful computation}\] 
\[f:X\to Y \qquad\qquad\quad\qquad f:X\to F(Y)\]
\end{framed}
% \begin{center}
% \begin{tabular}{ccc}
% 	pure function & vs & effectful computation \\
% 	$f:X\to Y$ && $c:X\to F(Y)$
% \end{tabular}
% \end{center}

典型例としては、例外処理 (exception handling) と呼ばれる計算的作用を表す関手 $F(Y)=Y+E$ を用いた、「作用つきの計算」$f:X\to Y+E$ が挙げられます。ここで $+$ は余積を表し、$E$ は ``エラーの集合'' とみなします。このとき、計算 $f:X\to Y+E$ は、単に入力値 $x\in X$ から出力値 $y\in Y$ を返すだけでなく、「もし例外が生じたらエラー $e\in E$ を返す」という副作用 (side effect) を持ちます。
% \[\begin{array}{cccc}f:&X&\longrightarrow&Y+E\\&\rotatebox{90}{$\in$}&&\rotatebox{90}{$\in$}\\&x&\longmapsto &y\\&x'&\longmapsto&e\end{array}\]
\[\begin{array}{cccccc}f:&X&\longrightarrow&Y\!\!&\!+\!&\!\!E\\&\rotatebox{90}{$\in$}&&&\!\!\rotatebox{90}{$\in$}\!\!\\&x&\longmapsto &y\!\!\\&x'&\longmapsto&&&\!\!e\end{array}\]
では、この例のような副作用を持つ計算(作用つき計算)の「合成」はどう考えればいいでしょうか。たとえば、2つの作用つき計算 $f:X\to Y+E$ と $g:Y\to Z+E$ を合成して、$g\circ f:X\to Z+E$ のようなものを作りたいとき、一筋縄ではいかないことがわかります。そこで、次節で導入する「モナド」と「クライスリ圏」という概念が役立ちます。\\%(次節でもあらためてこの例外処理の例について取り上げます。)

% なお、作用つき計算 $f:X\to F(Y)$ を、便宜的に $f:X\computationrightarrow Y$ という特殊な矢印で表記することがよくあります。(厳密に言えば、これはモナド $F$ のクライスリ圏における射を表す記法ですが、ここでは詳細の説明は省略します。この点については、次節を読んで「モナド」や「クライスリ圏」について学んだ後ならわかるはずです。)\\

\subsection{モナドとクライスリ圏}




\clearpage

\subsection{グロタンディーク・ファイブレーション入門}

さて本節では、次節以降で使う「グロタンディーク・ファイブレーション」をごく簡単に導入します。複雑に見えるかもしれませんが、これを使うと、述語変換子
やループ不変量
を「引き戻し」と「持ち上げ」によって統一的に書けます。

ここでは、まず直観を説明してから、後ほど正確な定義を述べます。例として $p:\Pred\to\Set$ というファイブレーションを考えてみましょう。ひとつずつステップを踏んで理解していきます。

\begin{enumerate}
	\item それぞれの対象 $X\in\Set$ に対して、「ファイバー」(fiber) と呼ばれる完備束 $\Pred_X\;(:=2^X)$ を割り当てます。
	\begin{center}
			\begin{tikzpicture}[auto,scale=0.55]
				\draw (-4,3) -- (-4,8) -- (-0.5,8) -- (-0.5,3) --cycle; % X
				\draw (0,1.5) -- (0,6.5) -- (3.5,6.5) -- (3.5,1.5) --cycle; % Z
				\draw (4,2.5) -- (4,7.5) -- (7.5,7.5) -- (7.5,2.5) --cycle; % Y
				%%%===
				\node[above] (predx) at (-2.25,8) {$\scriptstyle\Pred_X\;(=2^X)$};
				\node[above] (predz) at (1.75,6.5) {$\scriptstyle\Pred_Z\;(=2^Z)$};
				\node[above] (predy) at (5.75,7.5) {$\scriptstyle\Pred_Y\;(=2^Y)$};
				%=====
				\node (X) at (0,-0.5) {$X$};
				\node (Y) at (4,-0.5) {$Y$};
				\node (Z) at (2,-2.0) {$Z$};
				\draw [->] (X) -- node[above]{$\scriptstyle f$} (Y);
				\draw [->] (Y) -- node[below right]{$\scriptstyle g$} (Z);
				%=====
				\node (xphi) at (-2.5,3.5) {$\scriptstyle(\emptyset\subseteq X)$};
				\node (xpdash) at (-1.5,4.75) {$\scriptstyle(P'\subseteq X)$};
				\node (xp) at (-3,6) {$\scriptstyle(P\subseteq X)$};
				\node (xx) at (-2.5,7.5) {$\scriptstyle(X\subseteq X)$};
				\draw[->] (xphi) -- (xpdash);
				\draw[->] (xphi) -- (xp);
				\draw[->] (xpdash) -- (xp);
				\draw[->] (xp) -- (xx);
				%===
				\node (zphi) at (1.5,2) {$\scriptstyle(\emptyset\subseteq Z)$};
				\node (zpdash) at (2.5,3.25) {$\scriptstyle(R'\subseteq Z)$};
				\node (zp) at (1.0,4.5) {$\scriptstyle(R\subseteq Z)$};
				\node (zz) at (1.5,6) {$\scriptstyle(Z\subseteq Z)$};
				\draw[->] (zphi) -- (zpdash);
				\draw[->] (zphi) -- (zp);
				\draw[->] (zpdash) -- (zp);
				\draw[->] (zp) -- (zz);
				%===
				\node (yphi) at (5.5,3) {$\scriptstyle(\emptyset\subseteq Y)$};
				\node (ypdash) at (6.5,4.25) {$\scriptstyle(Q'\subseteq Y)$};
				\node (yp) at (5,5.5) {$\scriptstyle (Q\subseteq Y)$};
				\node (yy) at (5.5,7) {$\scriptstyle(Y\subseteq Y)$};
				\draw[->] (yphi) -- (ypdash);
				\draw[->] (yphi) -- (yp);
				\draw[->] (ypdash) -- (yp);
				\draw[->] (yp) -- (yy);
				%===
				\node (xbottom) at (-2.5,3) {};
				\node (zbottom) at (1.5,1.5) {};
				\node (ybottom) at (5.5,2.5) {};
				\draw (xbottom) to[bend right=15pt] (X);
				\draw[cross] (zbottom) to[bend right=18pt] (Z);
				\draw (ybottom) edge[bend left=16pt] (Y);
			\end{tikzpicture}
		\end{center}
	\item それぞれの射 $f:X\to Y$ に対して、$f^\ast:\Pred_Y\to\Pred_X$ を定めます。
	% \begin{small}
		\[
		\begin{array}{cccc}
			f^\ast:&2^Y&\longrightarrow&2^X\\[0pt]&\rotatebox{90}{$\in$}&&\rotatebox{90}{$\in$}\\[-2pt]&(Y\xrightarrow{g} 2)&{\;\longmapsto\;}&(X\xrightarrow{f}Y\xrightarrow{g} 2)
		\end{array}
		\] 
		% この $f^\ast$ が、$f$ による述語変換子にあたるものです。(後ほど説明します。)
		直観的には、以下の図のようなものだと理解するのがよいでしょう。
		\begin{center}
			\begin{tikzcd}
				\scriptstyle\Pred_X & \scriptstyle\Pred_Y \\[-22pt]
				\;\fbox{$\substack{~~~~~~\\~~~~\\~~~~\\P'~~~~~~\\[-2pt]~~~~~~~P''\\~\nwarrow~~~~~\\[-4pt]~~~~~\nearrow~~\\[3pt]~~P~~~\\~~~}$}\; \ar[d,dash,bend right=12pt]  & \;\fbox{$\substack{~~~~~~\\~~~~\\~~~\\~~~~~~Q'~\\~~~~~\\~~\nearrow~~\\~~~\\~~Q~~~~~\\~~~}$}\; \ar[d,dash,bend left=12pt] \ar[l,"f^\ast"'] \\[5pt]
				X \ar[r,"f"] & Y
			\end{tikzcd}
		\end{center}
	% \end{small}
	\item 以上で与えられたデータ $\Bigl((\Pred_X)_{X\in\Set}\,,\,(f^\ast:\Pred_Y\to\Pred_X)_{f:X\to Y\,\textrm{in}\,\Set}\Bigr)$ をもとにして、各ファイバーを貼り合わせて、「全体圏」(total category) と呼ばれる圏 $\Pred$ をつくります。
	\begin{small}
		\begin{center}
			\begin{tikzcd}
				\Pred \ar[d,"p"'] & \fbox{$\!\!\!\!\!\!\!\substack{~~~~~~~~~\\~~~~~\\~~~~~\\P'~~~~~~\\[-2pt]~~~~~P''~~~\\[-4pt]~~~~~~~~~~~~~~f^\ast \!(Q)\\[-2pt]~\rotatebox{130}{$\scriptstyle\longrightarrow$}~~~~~\\[-10pt]~~~~~~~~\rotatebox{98}{$\scriptstyle\longrightarrow$}~\nearrow~~~~\\[3pt]~~~~~~~P~~~~\\~~~~~~~~~~~~~~~~~}~\!\!\!\!\substack{~~\\~~\\~~\\~~\\\rotatebox{-20}{~$\longrightarrow$}\\~~\\\rotatebox{20}{$\longrightarrow$~}}\!\!\!\!\!\!\!\!\substack{\vdots\\[-4pt]\vdots\\[-4pt]\vdots\\[-4pt]\vdots\\[-4pt]\vdots\\~}~~~\substack{~~~~~~\\~~~~\\~~~\\~~~~~~Q'~\\~~~~~\\~~\nearrow~~\\~~~\\~~Q~~~~~\\~~~\\~~~\\~~~~~}~~~~$}\\
				\Set & X\xrightarrow{\qquad f \qquad} Y 
			\end{tikzcd}
		\end{center}
	\end{small}
	\item この $\Pred$ は、対象が組 $(X,\,P\subseteq X)$ であり、射が写像 $(X,\,P\subseteq X)\to(Y,\,Q\subseteq Y)$ であるような圏です。
	\begin{itemize}
		\item 対象: \[\left|\Pred\right|=\coprod_{X\in\Set}\left|\Pred_X\right|\]
		\item 射: 
			% \begin{small}
				\begin{prooftree}
					\AXC{$(X,\,P\subseteq X)\longrightarrow (Y,\,Q\subseteq Y)$ in $\Pred$}
					\doubleLine
					\UIC{$X\xrightarrow{\;f\;}Y$ in $\Set$ s.t. $P\sqsubseteq f^\ast (Q)$ in $\Pred_X$}
				\end{prooftree}
				% \end{small}
		\end{itemize}\vspace{5pt}
	\item 以下の条件を満たす関手 $p:\Pred\to\Set$ は「ファイブレーション」(fibration) と呼ばれます。(詳細は後述)
	\begin{small}
		\begin{center}
			\!\!\!\!\!\!
			\begin{tikzcd}[row sep=-2pt,column sep=11.5pt]
				 & &&&&&& R \ar[rrd,bend left=15pt,"\forall\,r"] \ar[rd,dashed,"{}{\exists !\,} s\!\!\!"',shorten >= -7pt] \ar[ddd,mapsto] && \\
				\Pred \ar[ddd,"p"'] &&Q \ar[ddd,mapsto] && ~{}^{\exists\,}f^\ast(Q)~ \ar[r,"\exists\,\overline{f}(Q)~"',shorten=-4pt] \ar[ddd,mapsto] & ~Q~ \ar[ddd,mapsto] &&& ~f^\ast(Q)~ \ar[r,"\overline f(Q)"',shorten=-3pt] \ar[ddd,mapsto] & ~Q~ \ar[ddd,mapsto]\\
				&&&\hspace*{-8pt}\implies\hspace*{-8pt} &&& \hspace*{-5pt}\textrm{s.t.}\hspace*{-5pt} &&&\\
				&&&&&&& p(R) \ar[rd,"\forall\,g"'] \ar[rrd,bend left=15pt,pos=0.65,"g\circ f"] && \\
				\Set & \,X\, \ar[r,"f"',shorten=-3pt] & \,Y\, && X \ar[r,"f"'] & Y &&& X \ar[r,"f"'] & Y
			\end{tikzcd}
			\!\!\!\!\!\!
		\end{center}
	\end{small}
	\item ここで、$\Pred$ を「全体圏」といい、$\Set$ を「基底圏」といいます。\\ また、$\overline{f}(Q):f^\ast(Q)\to Q$ を、$f$ の $Q$ による「カルテシアン持ち上げ」(cartesian lifting) といいます。\\ そして、$f^\ast$ を、$f$ の「置換」(substitution) といい、対象 $P,Q\in\Pred$ で $pP=pQ=Y$ なるものと、$\Pred$ の射 $(P\xrightarrow{u}Q)$ で $pu=\id_Y$ なるものについて、$\overline f_Q\circ f^\ast u=u\circ\overline f_P$ を満たすものと定められています。
	\begin{center}
		\begin{tikzcd}
				& f^\ast P \ar[d,"f^\ast u"'] \ar[r,"\overline f_P"] & P \ar[d,"u"] \\
			\Pred \ar[d,"p"'] & f^\ast Q \ar[r,"\overline f_Q"'] & Q \\
			\Set &X \ar[r,"f"'] & Y
		\end{tikzcd}
	\end{center}
	% \clearpage
	\item ここで、$(-)^*$ や $\overline{(-)}$ は ``関手的'' (functorial)、つまり恒等射と合成に整合します。
	\begin{small}
	\begin{align*}
		\begin{array}{rlcrl}
			\id_Y^\ast Q & = Q, & \quad\qquad & (g\circ f)^\ast(Q) & =f^\ast(g^\ast Q), \\[3pt]
			\overline{\id_Y}(Q) & = \id_Q, & & \overline{g\circ f}(Q) &=\overline gQ\circ\overline f(g^\ast Q).
		\end{array}
	\end{align*}
	\end{small}
	\item 自己関手 $F:\Set\to\Set$ と、ファイブレーション $p:\Pred\to\Set$ について、\\ 関手 $\dot{F}:\Pred\to\Pred$ が、$p$ に沿った $F$ の「持ち上げ」(fibred lifting) であるとは、
	\begin{center}
		\begin{small}
			\begin{tikzcd}[row sep=18pt]
				\Pred \ar[r,"\dot F"] \ar[d,"p"'] & \Pred \ar[d,"p"] \ar[ld,"\rotatebox{45}{$=$}",phantom]\\
				\Set \ar[r,"F"'] & \Set
			\end{tikzcd} 
			\quad かつ \quad
			\begin{tikzcd}[column sep=28pt,row sep=18pt]
				\Pred_Y \ar[r,"f^\ast"] \ar[d,"\dot F"'] & \Pred_X \ar[d,"\dot F"] \ar[ld,phantom,"\rotatebox{40}{$\cong$}"',shorten=12pt] \\
				\Set_{FY} \ar[r,"(Ff)^\ast\;"'] & \Set_{FX}
			\end{tikzcd}
		\end{small}
	\end{center} 
	を満たすことです。特に余代数との関わりの中では、以下の図のような状況を考えることが多いです。基底の関手 $F$ をどうやって $\dot F$ に持ち上げるかが重要になります。(後述)
	\begin{small}
		\begin{center}
			\begin{tikzcd}
				\Pred \ar[loop left,"\dot F"] \ar[d,"p"'] & \Pred_{X} \ar[rr,bend left=12pt,"\dot F_X"] && \Pred_{\,FX} \ar[ll,bend left=12pt,"c^\ast"] \\
				~\Set~ \ar[loop left,"F"]  & X \ar[rr,"c"'] && FX 
			\end{tikzcd}
		\end{center}
	\end{small}
	% \item 
\end{enumerate}

% \begin{center}
% 	\hrulefill
% \end{center}
\clearpage

\noindent 以下、ファイブレーションの正確な定義を念のため書いておきます。(ですが、上記で扱った $p\colon\Pred\to\Set$ の考え方だけをきちんと丁寧に押さえておけば、次節以降の内容を理解するうえで支障はありません。)

\begin{definition}[カルテシアン]
	$p:\E\to\B$ を関手とします。$\E$ の射 $(X\xrightarrow{f}Y)$ が $p$-カルテシアン ($p$-cartesian) であるとは、すべての $Z\in\E$ について、以下のプルバック図式が可換になることである。
	\begin{multicols}{2}
		\begin{small}
		\begin{center}
			\begin{tikzcd}[row sep=30pt,column sep=35pt]
				\E(Z,X) \ar[r,"f\circ(-)"] \ar[d,"p_{ZX}"']  \arrow[dr, phantom, "\lrcorner", very near start] & \E(Z,Y) \ar[d,"p_{ZY}"] \\
				\B(pZ,pX) \ar[r,"pf\circ(-)"'] & \B(pZ,pY) 
			\end{tikzcd}
		\end{center}
		\end{small}
		% ~\\
		\begin{small}
		\begin{center}
			\quad i.e. \quad 
			\begin{tikzcd}
				\exists!\,h \ar[r,mapsto] \ar[d,mapsto] & \forall\, g \ar[d,mapsto]\\
				\forall\,u \ar[r,mapsto] & \substack{\displaystyle ~~~~pg \\ \displaystyle\rotatebox{45}{$=~$}\\\displaystyle pf\circ u~~~~~~\\[3pt]\displaystyle~~~}
			\end{tikzcd}
		\end{center}
		\end{small}
	\end{multicols}
\end{definition}

\begin{remark}
	要するに以下と同じ。
	\begin{small}
	\begin{center}
		\begin{tikzcd}[row sep=3pt,column sep=20pt]
			& Z \ar[rrd,bend left=15pt,"\forall\,g"] \ar[rd,dashed,"{}{\exists !\,} h\!\!\!"']  && \\
			~\E~ \ar[ddd,"p"'] && X \ar[r,"f"']  & Y \\
			&&\\[5pt]
			& pZ \ar[rd,"\forall\,u"'] \ar[rrd,bend left=15pt,pos=0.65,"pg"] && \\
			~\B~ && pX \ar[r,"pf"'] & pY
		\end{tikzcd}
	\end{center}
	\end{small}
\end{remark}

\begin{definition}[ファイブレーション]
	関手 $p:\E\to\B$ がファイブレーション (fibration) であるとは、$\B$ の任意の射 $u:I\to J$ と、$J$ の上にある(つまり $pY=J$ であるような)任意の対象 $Y\in\E$ に対し、$\E$ の $p$-カルテシアン射 $f:X\to Y$ が $u$ の上に存在する(つまり $pf=u$である)ことである。
\end{definition}

\begin{remark}
	要するに以下と同じ。
	\begin{small}
	\begin{center}
		\begin{tikzcd}[column sep=35pt,row sep=16pt]
			\E \ar[d,"p"'] & & Y \ar[rd,"\implies",phantom] & \exists\, X \ar[r,"\exists\,f"',"\textrm{cartesian~\,}"] & Y \\
			\B & I \ar[r,"u"'] & J & I \ar[r,"u"'] & J
		\end{tikzcd}
	\end{center}
	\end{small}
\end{remark}

\begin{definition}[ファイバー]
	$p:\E\to\B$ を関手とする。各対象 $I\in\B$ について、圏 $\E_I$ をファイバー (fiber) と呼ぶ。対象は、$I$ の上にある(つまり $pX=I$ を満たすような)$X\in\E$ であり、射は $\id_I$ の上にある(つまり$pf=\id_I$ を満たすような) $\E$ の射 $f:X\to Y$ である。
\end{definition}

% \clearpage

\begin{definition}[置換]
	$p:\E\to\B$ をファイブレーションとし、$u:I\to J$ を $\B$ の射とします。$u$ の置換 (substitution) とは、関手 $u^\ast\colon\E_J\to\E_I$ であって以下を満たすものである。(ここで $\overline u_X\colon u^\ast X\to X$ および $\overline{u}_Y\colon u^\ast Y\to Y$ は、$p$-カルテシアン射でかつ $p(\overline u_X)=u$ および $p(\overline u_Y)=u$ を満たすものとする。)\vspace{3pt}
	\begin{center}
		\begin{tikzcd}%[row sep=15pt]
			u^*X \ar[d,"u^*f"',dashed] \ar[r,"\overline u_X"] & X \ar[d,"f"] \\ u^*Y \ar[r,"\overline u_Y"] & Y \\[-8pt]
			I \ar[r,"u"] & J
		\end{tikzcd}
	\end{center}
	\begin{itemize}
		\item 対象 $X\in\E_J$ に対し、対象 $u^\ast X\in\E_I$ を返し、
		\item $\E_J$ の射 $f:X\to Y$ に対し、$\E_I$ の射 $u^\ast f:u^\ast X\to u^\ast Y$ を一意に定め、以下のプルバック図式を満たす: 
		\begin{center}
		\begin{small}
			\begin{tikzcd}[column sep=38pt,row sep=30pt]
				% \E(u^\ast X,u^\ast Y) \ar[r,"{\E(u^{\!*\!} X,\overline{u}_Y)}"] \ar[d,"p\circ(-)"'] \arrow[dr, phantom, "\lrcorner", very near start] & \E(u^\ast X,Y) \ar[d,"p\circ(-)"] \\
				\E(u^\ast X,u^\ast Y) \ar[r,"\overline{u}_Y\circ(-)"] \ar[d,"p"'] \arrow[dr, phantom, "\lrcorner", very near start] & \E(u^\ast X,Y) \ar[d,"p"] \\
				% \B(I,I) \ar[r,"{\B(I,u)}"'] & \B(I,J)
				\B(I,I) \ar[r,"u\circ(-)"'] & \B(I,J)
			\end{tikzcd}
		% \end{small}
		\qquad\quad i.e. \quad 
		% \begin{small}
			\begin{tikzcd}[column sep=32pt,row sep=25pt]
				\exists !\,u^\ast f \ar[r,mapsto,"\textrm{\ajMaru{2}}"] \ar[d,mapsto,"\textrm{\ajMaru{1}}"'] & f\circ\overline{u}_X \ar[d,mapsto] \\ 
				\id_I \ar[r,mapsto] & u 
			\end{tikzcd}
		\end{small}
		\end{center}
		すなわち、\textrm{\ajMaru{1}} $p(u^\ast f)=\id_I$ と \textrm{\ajMaru{2}} $f\circ\overline{u}_X=\overline{u}_Y\circ u^\ast f$ を満たす。
	\end{itemize}
\end{definition}

\begin{remark}
	同一の射の上にあるカルテシアン射は ``一意な同型を除いて一意'' (unique up to unique iso) である。より正確に言うと、ファイブレーション $p:\E\to\B$ と、$\B$ の射 $u:I\to J$ が与えられ、$Y\in\E_J$ があり、2つの $p$-カルテシアン射 $\overline{u}_Y:u^\ast Y\to Y$ と $f:X\to Y$ がいずれも $u$ の上にあるとき、一意な同型射 $\alpha:X\xrightarrow{\cong} u^\ast Y$ がファイバー $\E_I$ に存在し、$f=\overline u_Y\circ\alpha$ が成り立つ。(証明略)%\vspace{3pt}

	\begin{small}
	\begin{center}
		\begin{tikzcd}
			& X \ar[r,"f"] \ar[d,shift right=5pt,"~\rotatebox{-90}{$\cong$}",dashed,phantom] & Y \ar[d,"\rotatebox{90}{$=$}",phantom] \\[-5pt]
			\E \ar[d,"p"'] & u^\ast Y \ar[r,"\overline u_Y"'] & Y \\[-3pt]
			\B & I \ar[r,"u"] & J
		\end{tikzcd}
	\end{center}
	\end{small}
\end{remark}

\begin{proposition} \label{prop:gamma-nat-trans}
	$p:\E\to\B$ をファイブレーションとする。$F:\B\to\B$, $\dot F:\E\to\E$ をそれぞれ $\B, \E$ の自己関手とし、$p\circ\dot F=F\circ p$を満たすものとする。
	\begin{small}
	\begin{center}
		\begin{tikzcd}[column sep=16pt,row sep=16pt]
			\E \ar[r,"\dot F"] \ar[d,"p"'] & \E \ar[d,"p"] \\
			\B \ar[r,"F"'] & \B
		\end{tikzcd}
	\end{center}
	\end{small}
	このとき、$\B$ の任意の射 $(I\xrightarrow{u}J)$ に対し、以下の自然変換 $\gamma:\dot F\circ u^\ast\Rightarrow(Fu)^\ast\circ\dot F$ が存在する。\vspace{-5pt}
	\begin{multicols}{2}
		\begin{small}
		\begin{center}
			~\\[-2pt]
			\begin{tikzcd}[column sep=32pt,row sep=25pt]
				\E_J \ar[r,"u^\ast"] \ar[d,"\dot F"'] & \E_I \ar[d,"\dot F"] \ar[ld,Rightarrow,"\gamma"',shorten=12pt] \\
				\E_{FJ} \ar[r,"(Fu)^\ast"'] & \E_{FI}
			\end{tikzcd}
		\end{center}
		\end{small}
		%
		\begin{small}
		\begin{center}
			i.e.\quad 
			\begin{tikzcd}[row sep=10pt]
				X \ar[r,mapsto,"u^\ast"] \ar[dd,mapsto] & u^\ast X \ar[d,mapsto] \\
				& \dot F(u^\ast X) \ar[d,"\gamma_X"]\\
				\dot FX \ar[r,mapsto] & (Fu)^\ast \dot FX
			\end{tikzcd}
		\end{center}
		\end{small}
	\end{multicols}
\end{proposition}

\begin{proof}
	$X\xmapsto{\,(pX=J)\,}J$ および $\dot FX\xmapsto{\,(p\dot FX=FpX=FJ)\,}FJ$ と考えると
	\begin{small}
	\begin{center}
		\begin{tikzcd}
			\E \ar[loop left, "\dot F"] \ar[d,"p"'] & u^\ast X \ar[r,"\overline u"] & X \\
			\B \ar[loop left, "F"] & I \ar[r,"u"'] & J  
		\end{tikzcd}
		\;$\longmapsto$\;
		\begin{tikzcd}
			\dot F(u^\ast X) \ar[r,"\dot F\,\overline u"] & \dot FX \\
			FI \ar[r,"u"'] & FJ  
		\end{tikzcd}
	\end{center}
	\end{small}
	が得られる。それゆえ、
	\begin{small}
	\begin{center}
		\begin{tikzcd}
			& \dot F(u^\ast X) \ar[rd,"\dot F\,\overline u",bend left=12pt] \ar[d,dashed,"\gamma_X"'] &  \\ 
			\E \ar[d,"p"'] & (Fu)^\ast \dot FX \ar[r,"\overline{Fu}"'] & \dot FX \\
			\B & FI \ar[r,"Fu"'] & FJ
		\end{tikzcd}
	\end{center}%\vspace{-6pt}
	\end{small}
	となる。($\gamma$ の自然性の証明はここでは省略する。)
\end{proof}

\begin{definition}[カルテシアンを保つ] \label{def:preserving-cartesian}
	$p:\E\to\B$ をファイブレーションとする。関手 $\dot F:\E\to\E$ が $p$-カルテシアンを保つ (preserves $p$-cartesian) とは、$\dot F$ が各 $p$-カルテシアン射 $f$ を別の $p$-カルテシアン射 $\dot Ff$ に送ること。\vspace{5pt}%つまり、
	\begin{center}$(X\xrightarrow{f}Y)$ in $\E$ is $p$-cartesian $\implies$ $(\dot FX\xrightarrow{\dot Ff}\dot FY)$ in $\E$ is $p$-cartesian.\end{center}
\end{definition}

\begin{proposition}
	$p:\E\to\B$ をファイブレーションとする。$F:\B\to\B$, $\dot F:\E\to\E$ をそれぞれ $\B, \E$ の自己関手とし、$p\circ\dot F=F\circ p$を満たすものとする。(ここまでの状況は、先ほどの\textgt{命題\ref{prop:gamma-nat-trans}}のときと同じ。)

	\begin{small}
	\begin{center}
		\begin{tikzcd}[column sep=16pt,row sep=16pt]
			\E \ar[r,"\dot F"] \ar[d,"p"'] & \E \ar[d,"p"] \\
			\B \ar[r,"F"'] & \B
		\end{tikzcd}
	\end{center}
	\end{small}
	このとき、以下の同値関係が成り立つ。\vspace{5pt}
	\begin{center}
	$\dot F$ preserves $p$-cartesian $\iff$ $\gamma:\dot F\circ u^\ast\Rightarrow(Fu)^\ast\circ\dot F$ is isomorphic. \end{center}
\end{proposition}

\begin{proof}
	任意の $p$-カルテシアン射 $f:X\to Y$ をとり、以下の状況を考える。
	\begin{center}
		\begin{small}
			\begin{tikzcd}[column sep=32pt,row sep=28pt]
				\E_{pY} \ar[r,"(pf)^\ast"] \ar[d,"\dot F"'] & \E_{pX} \ar[d,"\dot F"] \ar[ld,Rightarrow,"\gamma"',shorten=12pt] \\
				\E_{FpY} \ar[r,"(Fpf)^\ast"'] & \E_{FpX}
			\end{tikzcd}
			\qquad i.e. \qquad 
			\begin{tikzcd}[row sep=12pt]
				Y \ar[r,mapsto] \ar[dd,mapsto] & X \ar[d,mapsto] \\
				& \dot FX \ar[d,"\gamma_Y"]\\[4pt]
				\dot FY \ar[r,mapsto] & (Fpf)^\ast \dot FY
			\end{tikzcd}
		\end{small}
	\end{center}
	言い換えると、状況は以下のとおり。\vspace{-5pt}
	\begin{center}
		\begin{small}
			\begin{tikzcd}
				&&&& \dot FX \ar[rd,bend left=15pt,"\dot Ff"] \ar[d,"\gamma_Y"',dashed] & \\
				\E \ar[d,"p"'] & X \ar[r,"\substack{\overline{pf}\\\rotatebox{90}{$\,=\,$}\\f}"] & Y \ar[rrd,"\longmapsto",phantom] && (Fpf)^\ast FY \ar[r,"\overline{Fpf}"'] & \dot FY\\[-5pt]
				\B & pX \ar[r,"pf"'] & pY && FpX \ar[r,"Fpf"'] & FpY
			\end{tikzcd}
		\end{small}
	\end{center} 
	ここで、$\dot Ff$ が $p$-カルテシアンである $\iff$ $\gamma_Y$ が同型射である。(なぜなら、$\dot Ff$ と $\overline{Fpf}$ はいずれも $Fpf$ の上にあり、しかも $\overline{Fpf}$ は定義からして $p$-カルテシアンであるので、($\Rightarrow$): もし $\dot Ff$ も $p$-カルテシアンなら、一意な同型射 $\gamma_Y$ が手に入るし、($\Leftarrow$): もし $\gamma_Y$ が同型射なら、$\dot Ff$ は $p$-カルテシアンである。)
	% Here, $\dot Ff$ is $p$-cartesian iff $\gamma_Y$ is isomorphic. (Because both of $\dot Ff$ and $\overline{Fpf}$ are above $Fpf$, and $\overline{Fpf}$ is $p$-cartesian by definition, thus ($\Rightarrow$): if $\dot Ff$ is also $p$-cartesian, then we have a unique isomorphism $\gamma_Y$; and ($\Leftarrow$): if $\gamma_Y$ is a isomorphism, then $\dot Ff$ is also $p$-cartesian.) 
\end{proof}

% \clearpage

\begin{definition}[関手の持ち上げ]
	$p:\E\to\B$ をファイブレーションとする。$F:\B\to\B$ を $\B$ における自己関手とする。このとき、$\dot F:\E\to\E$ が $p$ に沿った $F$ のファイバー持ち上げ (fibered lifting) であるとは、$p\circ \dot F=F\circ p$ を満たし、かつ $\dot F$ がカルテシアンを保つ(cf. \textgt{定義\ref{def:preserving-cartesian}})ことである。
\end{definition}

\begin{remark}
	すなわち、$\dot F$ は~
	\begin{small}
		\begin{tikzcd}
			\E \ar[r,"\dot F"] \ar[d,"p"'] & \E \ar[d,"p"] \ar[ld,"\rotatebox{45}{$=$}",phantom]\\
			\B \ar[r,"F"'] & \B
		\end{tikzcd} 
	\end{small}
	~と~
	\begin{small}
		\begin{tikzcd}[column sep=22pt]
			\E_J \ar[r,"u^\ast"] \ar[d,"\dot F"'] & \E_I \ar[d,"\dot F"] \ar[ld,phantom,"\rotatebox{40}{$\cong$}"',shorten=12pt] \\
			\E_{FJ} \ar[r,"(Fu)^\ast\;"'] & \E_{FI}
		\end{tikzcd}
	\end{small}
	を満たす。
	\begin{itemize}
		\item 前者より、各 $X\in\B$ について、$\dot F_X:\E_X\to\E_{FX}$ を定義できることが従う。
		\item 後者は、「関手の持ち上げが置換と整合的である」ということを意味する。
	\end{itemize}
\end{remark}

さて、次章からは、これらの道具立てをフル活用して、様々な種類の(非決定的、確率的、etc.)プログラムに対する最弱事前条件を定める方法を紹介します。

% \begin{remark}
% 	$\dot F_X:\E_X\to\E_{FX}$ である。
% \end{remark}

\clearpage

\section{最弱事前条件をどのように手に入れるか}

さて、本節では、非決定的プログラムや確率的プログラムなどといった様々な種類のプログラムに対する述語変換子意味論を得る方法について、圏論的な視座から考えます。一般に、状態の集合 $X, Y$ と真理値の集合 $\Omega$ が与えられたときに、効果付き計算 $c:X\to FY$ に基づいて、事後条件 $q:Y\to\Omega$ を最弱事前条件 $wp\llbracket c\rrbracket(q):X\to \Omega$ に変換するための関数である述語変換子 $wp\llbracket c\rrbracket:\Omega^Y\to\Omega^X$ を手に入れるために、どんな手続きを経たらよいでしょうか。

いくつか具体例を見てから、その後に一般論を考えましょう。

\subsection{具体例}

以下では、3つのケースについて、それぞれ具体例を考えていきます。

\subsubsection*{具体例 (1): 決定的な2値的モデル検査の場合}
% 以下のように考えます。
\vspace{10pt}

\begin{itemize}
	\item 真理値の集合を $\Omega=2=\{0,1\}$ とします。
	\item 考えるべき完備束 $L=\Omega^X$ は、以下の通りです。(ここでは各 $X$ を状態集合とします。)\begin{center}$L=2^X=\{q:X\to 2,\textrm{ predicates}\}$\\\qquad\qquad$\ni\,\bot \,(\textrm{everywhere false}),\top\,(\textrm{everywhere true})$ \end{center}
	\item ここで、$f:X\to Y$ という計算を考えます。 \\ (なお、$X=Y$ の場合、以下のような余代数です。)\vspace{5pt}\begin{center}\begin{small}$\left(\begin{array}{cccc} c:&X & \longrightarrow & X\\[-2pt] &\rotatebox{90}{$\in$} & & \rotatebox{90}{$\in$}\\[-4pt] & x & \mapsto & x' \end{array}\,\right)$\end{small} \qquad\quad \begin{small}\begin{tikzcd}\circled{$x_1$}\ar[r] & \circled{$x_2$} \ar[r] & \cdots\end{tikzcd}\end{small}\end{center} \vspace{3pt}
	\item $f:X\to Y$ に対し、述語変換子 $\left(\begin{array}{cccc}f^\ast:&2^Y&\longrightarrow & 2^X\\[-2pt]&\rotatebox{90}{$\in$}&&\rotatebox{90}{$\in$}\\[-2pt]& \scriptstyle(X\,\overset{q}{\to}\,2)&\mapsto & \scriptstyle (X\,\overset{f^\ast(q)}{\longrightarrow}\,2) \end{array}\right)$ を \vspace{10pt} \\ $f^\ast:2^Y\to 2^X$, $f^\ast(Q)=\{x\in X\mid f(x)\in Q\}$ で定めます。\vspace{5pt}
	\item いま、以下の関数を定めましょう。\vspace{3pt}\\ 
	% $\exists_c:2^X\to 2^Y$, $\exists_c(P)=\{f(x)\in Y\mid x\in P\}=\{y\in Y \mid \exists\, x\in X,\; f(x)=y\;\textrm{and}\;x\in P\}$
	$\begin{array}{lccl}\exists_f:2^X\to 2^Y,&\exists_f(P)&\!\!=\!\!&\{f(x)\in Y\mid x\in P\}\\&&\!\!=\!\!&\{y\in Y \mid \exists\, x\in X,\; f(x)=y\;\textrm{and}\;x\in P\}\\\forall_f:2^X\to 2^Y,& \forall_f(P)&\!\!=\!\!&\{y\in Y\mid f^\ast(\{y\})\subseteq P\}\\&&\!\!=\!\!&\{y\in Y\mid\forall\,x\in X,\;f(x)=y\;\textrm{implies}\;x\in P\}\end{array}$ \vspace{5pt}
	\item このとき、$\exists_f\dashv f^\ast\dashv \forall_f$ が成り立ちます。\\(これはまさに、トポスの基本定理 (the fundamental theorem of topos) として知られるものと同等です。)
	\begin{center}
				\begin{tikzcd}[column sep=30pt]
					% 2^X \ar[r,bend left=30pt,"\amalg_f","\bot"'] \ar[r,bend right=30pt,"\Pi_f"',"\bot"] & 2^Y \ar[l]
					2^X \ar[r,bend left=30pt,"\exists_f","\bot"'] \ar[r,bend right=30pt,"\forall_f"',"\bot"] & 2^Y \ar[l]
				\end{tikzcd}
			\end{center}
			%
			\begin{small}
				\begin{center}
				% \begin{prooftree}
					\AXC{$\exists_f(P)\sqsubseteq Q$ in $2^Y$} \doubleLine
					\UIC{$P\sqsubseteq f^\ast(Q)$ in $2^X$}
					\DisplayProof
					\; and \;
					\AXC{$f^\ast(Q)\sqsubseteq P$ in $2^X$} \doubleLine
					\UIC{$Q\sqsubseteq \forall_f(P)$ in $2^Y$}
					\DisplayProof
				% \end{prooftree}
				\end{center}
			\end{small}
\end{itemize}

\clearpage

\subsubsection*{具体例 (2): 非決定的な2値的モデル検査の場合}
\vspace{10pt}

\begin{itemize}
	\item 真理値の集合を $\Omega=2=\{0,1\}$ とします。
	\item 考えるべき完備束 $L=\Omega^X$ は、以下の通りです。 \begin{center}$L=2^X=\{q:X\to 2,\textrm{ predicates}\}$\\\qquad\qquad$\ni\,\bot \,(\textrm{everywhere false}),\top\,(\textrm{everywhere true})$ \end{center} (ここまでは一つ前の例と同じです。)\vspace{5pt}
	\item ここで、$f:X\to\mathcal PY$ という計算を考えます。\\(なお、$X=Y$ の場合、以下のような余代数です。)\vspace{5pt} %\\ 
	\begin{center}
	\begin{small}$\left(\begin{array}{cccccc} c:&X & \longrightarrow & \mathcal PX\\[-1pt] &\rotatebox{90}{$\in$} & & \rotatebox{90}{$\in$}\\[-2pt] &x & \mapsto & c(x)&=&\{x'\mid x\to x'\} \end{array}~\textrm{Kripke structure}\right)$\end{small} \qquad\quad 
	\begin{small}
		\begin{tikzcd}[row sep=3pt]
			& \circled{$\;y\;$}\\[-5pt]
			\circled{$\;x\;$} \ar[ru] \ar[r] \ar[rd] & \circled{$y'$}\\
			& {~\cdots~}
		\end{tikzcd}
	\end{small}
	\hspace*{25pt}
	\end{center}
	\item さて、述語変換子 $\left(\begin{array}{cccc}f^\ast:&2^Y&\longrightarrow & 2^X\\[-2pt]&\rotatebox{90}{$\in$}&&\rotatebox{90}{$\in$}\\[-2pt]& \scriptstyle(X\,\overset{q}{\to}\,2)&\longmapsto & \scriptstyle (X\,\overset{\!f^\ast(q)\!}{\longrightarrow}\,2) \end{array}\right)$ をどうやって手に入れればよいでしょうか? \vspace{15pt} \\ 以下のようにして得られます。(上段が入力、下段が出力です。)\vspace{5pt}
	\begin{prooftree}
		\AXC{$q:Y\to 2$,\; post-condition}
		\UIC{$\left(\begin{array}{cccccccc}f^\ast(q):&X&\xrightarrow{~f~}&\mathcal PY&\xrightarrow{\;\mathcal P q\;}&\mathcal P2&\xrightarrow[\textrm{modality}]{\tau}&2 \\[-6pt] & \rotatebox{90}{$\in$} && \rotatebox{90}{$\in$} && \rotatebox{90}{$\in$} && \rotatebox{90}{$\in$} \\[-4pt] & \scriptstyle x & \longmapsto & \scriptstyle \{x_1,\dots,\,x_n\} & \longmapsto & \scriptstyle \{q(x_1),\dots,\,q(x_n)\} & \longmapsto & \scriptstyle 0 \textrm{ or } 1 \end{array}\right)$} 
	\end{prooftree} \vspace{5pt}
	\item $f:X\to\mathcal PY$ に対し、\\ 述語変換子 $f^\ast$ は、様相 $\tau:\mathcal P2\to 2$(パラメーターマップとも呼ばれる)の選び方に拠って決まります。\\ $f^\ast$ を以下のように2通りの異なるしかたで定めることができます。 \\$\Box_f(-):2^Y\to 2^X,\,q\mapsto\{x\in X\mid \forall y\in Y\,(x\to y\textrm{ implies } y\in q)\}$ \\ $\Diamond_f(-):2^Y\to 2^X,\,q\mapsto\{x\in X \mid \exists y\in Y \, ( x\to y \textrm{ and } y\in q)\}$ \vspace{5pt}
	\item いま、以下の関数を定めましょう。\\ $\exists_f:2^X\to 2^Y$, $p\mapsto \{y\in Y\mid \exists \, x\in X,\,x\to y \textrm{ and } x\in p\}$ \\ $\forall_f:2^X\to 2^Y$, $p\mapsto \{y\in Y\mid \forall \, x\in X,\,x\to y \textrm{ implies } x\in p\}$ \vspace{5pt}
	\item このとき、$\exists_f\dashv \Box_f$ および $\Diamond_f\dashv\forall_f$ がともに成り立ちます。
	\begin{center}
		\begin{tikzcd}[column sep=30pt]
			2^X \ar[r,bend left=30pt,"\exists_f","\bot"']  & 2^Y \ar[l,"\Box_f"]
		\end{tikzcd}
		\qquad\qquad
		\begin{tikzcd}[column sep=30pt]
			2^X  \ar[r,bend right=30pt,"\forall_f"',"\bot"] & 2^Y \ar[l,"\Diamond_f"']
		\end{tikzcd}
	\end{center} \vspace{0pt}
	%
	\begin{small}
	\begin{center}
		% \begin{prooftree}
		\AXC{$\exists_f(P)\sqsubseteq Q$ in $2^Y$} \doubleLine
		\UIC{$P\sqsubseteq \Box_f(Q)$ in $2^X$}
		\DisplayProof
		\; and \;
		\AXC{$\Diamond_f(Q)\sqsubseteq P$ in $2^X$} \doubleLine
		\UIC{$Q\sqsubseteq \forall_f(P)$ in $2^Y$}
		\DisplayProof
		% \end{prooftree}
		\end{center}
	\end{small}
\end{itemize}

\clearpage

\subsubsection*{具体例 (3): 確率的モデル検査の場合}
\vspace{10pt}

\begin{itemize}
	\item 真理値の集合を $\Omega=[0,1]$ とします。
	\item 考えるべき完備束 $L=\Omega^X$ は、以下の通りです。 \begin{center}$L=[0,1]^X=\{q:X\to [0,1],\;\textrm{predicates as random variables}\}$\end{center} 
	\item ここで、$f:X\to\mathcal DY$ という計算を考えます。\\(なお、$X=Y$ の場合、以下のような余代数です。)
	\begin{center}
	$\Bigl(c:X\to\mathcal DX=\{d:X\to [0,1],\;\textrm{$X$ 上の確率分布}\}\Bigr)$
	\qquad
	\begin{small}
		% \begin{center}
			\begin{tikzcd}[row sep=8pt]
				& \circled{$x_1$} \\[-3pt]
				\circled{$\;x\;$} \ar[ru,"p_1"] \ar[r,"p_2"] \ar[rd,"\cdots"'] & \circled{$x_2$} \\
				& {~\cdots~}
			\end{tikzcd}
		% \end{center}
	\end{small}
	\hspace*{50pt}
	\end{center}
	\item さて、述語変換子 $f^\ast:[0,1]^Y\to[0,1]^X$ をどうやって手に入れればよいでしょうか?\vspace{2pt} \\ 以下のようにして得られます。(上段が入力、下段が出力です。)\vspace{5pt} 
	\begin{prooftree}
		\AXC{$q:Y\to[0,1]$,\; post-condition}
		\UIC{$\left(\begin{array}{cccccccc} f^\ast(q): & X & \xrightarrow{~f~} & \mathcal DY & \xrightarrow{\;\mathcal Dq\;} & \mathcal D[0,1] & \xrightarrow[=Av]{\tau}& [0,1] \\[-3pt] &\rotatebox{90}{$\in$} && \rotatebox{90}{$\in$} && \rotatebox{90}{$\in$} && \rotatebox{90}{$\in$} \\[-3pt] &\scriptstyle x & \longmapsto & \scriptstyle \sum p_i\Ket{x_i} & \longmapsto & \scriptstyle \sum p_i\Ket{q(x_i)} & \longmapsto & \scriptstyle \sum p_i\cdot q(x_i) \end{array}\right)$}
	\end{prooftree} \vspace{5pt}
	\item $f:X\to\mathcal PY$ に対し、述語変換子 $f^\ast:[0,1]^Y\to[0,1]^X$ は以下で定義されます。 \[f^\ast(q)(x)=\sum_{x'\in X}f(x)(x')\cdot q(x')\] 
	つまり、$f^\ast(q)$ は、各 $x$ に対し、$q$ による値の期待値 $p_1 \cdot q(x_1) + p_2 \cdot q(x_2) + \cdots$ を割り当てます。
	\begin{center}	
		\begin{small}
			\begin{tikzcd}[row sep=8pt]
				& \circled{$x_1$} \\[-3pt]
				\circled{$\;x\;$} \ar[ru,"p_1"] \ar[r,"p_2"] \ar[rd,"\cdots"'] & \circled{$x_2$} \\
				& {~\cdots~}
			\end{tikzcd}
		\end{small}
	\end{center}
	\item なお、この $f^\ast$ は、左右いずれの随伴も持ちません。
	\\ (対応策としては、米田埋め込みを使う方法があります。詳細は省略しますが、おおむね以下の図の通りです。)
	\begin{center}
		\begin{tikzcd}[column sep=68pt,row sep=40pt]
			\widehat L \ar[rd,shift right=2.0pt,"\Lan_y(h)"'] \ar[rd,dash,shift left=0.8pt,phantom,"\rotatebox{55}{$\scriptstyle\dashv$}"]\\
			L \ar[u,"y"]\ar[r,"h"'] & L' \ar[ul, shift right=4.45pt, "\Ran_h(y)"']
		\end{tikzcd}
		\qquad\qquad 
		\begin{tikzcd}[column sep=58pt, row sep=30pt]
			\widehat L \ar[r,"\Lan_y(y\,\circ\, f^\ast)"'] & \widehat L \ar[l,bend right=22pt,"\exists g"',"\top"] \\
			L \ar[r,"f^\ast"'] \ar[u,"y"] & L \ar[u,"y"']
		\end{tikzcd}\vspace{8pt}\\
			$\Lan_y(h)(P)=\bigsqcup_{x\in P}h(x),\qquad \Lan_y(y\circ f^\ast)(P)=\{z\in L\mid\exists\,x\in P,\;z\sqsubseteq f^\ast(x)\}$
	\end{center}
\end{itemize}
\vspace{5pt}

\clearpage

\subsection{一般論}

\noindent 
これまで見た具体例をふまえて、一般論を展開していきましょう。

\begin{itemize}
	\item 副作用つき計算 $c:X\to FY$、真理値の集合 $\Omega$、様相 $\tau:F\Omega\to\Omega$ が与えられたとき、\\ 述語変換子 $wp\llbracket c\rrbracket:\Omega^Y\to\Omega^X$ は以下のしかたで得られます。
	\begin{prooftree}
		\AXC{$Y\xrightarrow{~q~} \Omega$,\; post-condition}
		\UIC{$wp\llbracket c \rrbracket (q):X\xrightarrow{~c~}FY\xrightarrow{\;Fq\;}F\Omega\xrightarrow{~\tau~}\Omega$,\; weakest pre-condition}
	\end{prooftree}\vspace{5pt}
	このように $wp\llbracket c\rrbracket (q)=\tau\circ Fq\circ c$ で定まります。\vspace{8pt} %\clearpage
\end{itemize}

\noindent 
これを、グロタンディーク・ファイブレーションの言葉で次のように捉えることができます。

\begin{itemize}
	\item (状況設定) ファイブレーション $p:\Pred\to\Set$ を考えます。各ファイバー $\Pred_X=\Omega^X$ は「各状態集合 $X$ 上の述語全体」であり順序集合をなし、各 $f:X\to Y$ に対する置換 $f^\ast:\Omega^Y\to\Omega^X,\,q\mapsto q\circ f$ は「事後条件 $q$ を $f$ によって引き戻した事前条件」を表します。 \vspace{10pt}
	\item (課題) いま、副作用つき計算$c:X\to FY$が与えられたとき、単に $c^\ast$ を考えると、これは $\Omega^{FX}\to\Omega^X$という型を持ちます。しかし、われわれが欲しい述語変換子は、$\Omega^Y\to\Omega^X$ という型を持つものです。\\(というのも、事後条件($Y$上の述語)から事前条件($X$上の述語)への変換を行いたいからです。)\vspace{10pt}
	\item (解決策) ここで、様相 $\tau:F\Omega\to\Omega$ が重要な役割を果たします。この $\tau$ は、「\uwave{$Y$上の述語}」から「\uwave{$FY$上の述語}」へと持ち上げる操作、\textbf{述語持ち上げ} (predicate lifting) を誘導します。\vspace{8pt}\\
	具体的には、各 $Y$ について、\[\dot F_Y:\Omega^Y\to\Omega^{FY},\quad \dot F_Y(q):=\tau\circ Fq\qquad \textrm{(ただし $q:Y\to\Omega$ を任意の述語とする)}\] と定めます。この $\dot F_Y$ は、「$Y$上の述語$q$を、計算的作用$F$で包まれた状態空間$FY$上の述語 $\dot F_Y(q)$ へと持ち上げる」操作です。これを、\textbf{述語持ち上げ}と呼びます。(通常、$\tau$ には単調性を課し、$\dot F_Y$ が単調になるようにします。)\vspace{8pt} \\
	すると、最弱事前条件 $wp\llbracket c\rrbracket$ は、この「述語持ち上げ $\dot F_Y$」と「計算 $c$ による引き戻し $c^\ast$」の合成として書けます。
	\[\begin{tikzcd}
		\Omega^Y \ar[r,"\dot F_Y"] \ar[rd,"wp\llbracket c\rrbracket"'] & \Omega^{FY} \ar[d,"c^\ast"]\\
		& \Omega^X
	\end{tikzcd}\]
	つまり、任意の事後条件 $q:Y\to\Omega$に対して
	\begin{align*}
		wp\llbracket c\rrbracket(q)&=c^\ast(\dot F_Y(q))\\ &=c^\ast(\tau\circ Fq) \\ &=\tau\circ Fq\circ c
	\end{align*}
	となります。まとめると、述語変換子は %\vspace{1pt}
	\begin{enumerate}
		\item[1)~~] $Y$ 上の述語 $q$ を、\uwave{様相 $\tau$ を用いて} $FY$ 上に\kenten{持ち上げて}、%\vspace{1pt}
		\item[2)~~] それを \uwave{計算 $c$ に沿って} $X$ 上に\kenten{引き戻す}、%\vspace{2pt}
	\end{enumerate}
	という2段階のプロセスとして理解できます。(スローガン:「\textbf{持ち上げて、引き戻す!}」)
\end{itemize}


\clearpage

\section{まとめ}



\clearpage

%-----
% \clearpage
\begin{thebibliography}{99}
    \item 
\end{thebibliography}
%-----


\end{document}